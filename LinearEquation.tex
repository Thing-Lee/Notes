\documentclass{article}
\usepackage[UTF8]{ctex}
\usepackage{amsmath}
\usepackage{amssymb}
\usepackage{geometry}%编辑页面边距
\usepackage[dvipsnames,svgnames]{xcolor}
\usepackage[strict]{changepage} % 提供一个 adjustwidth 环境
\usepackage{framed} % 实现方框效果
\usepackage{ctex}
\usepackage[backref]{hyperref}%目录链接
\usepackage{pifont}%编辑带圈数字
\usepackage{fancyhdr}%添加页眉
\pagestyle{fancy}%页眉
\fancyhead[R]{Edit By LiShiying}%页眉
\hypersetup{colorlinks=true,linkcolor=black,anchorcolor=megenta,citecolor=yellow}
\geometry{right=2.5cm,left=2.5cm,top=2.5cm,bottom=2.5cm}


\newtheorem{warning}{}  
\definecolor{washade}{rgb}{0.99,0.95,0.94} % 方框底部颜色
\newenvironment{wa}{%
\def\FrameCommand{%
\hspace{1pt}%
{\color{LightCoral}\vrule width 2pt}%竖线颜色
{\color{washade}\vrule width 4pt}%
\colorbox{washade}%
}%
\MakeFramed{\advance\hsize-\width\FrameRestore}%
\noindent\hspace{-4.55pt}% disable indenting first paragraph
\begin{adjustwidth}{}{7pt}%
\vspace{2pt}\vspace{2pt}%
\normalfont %
}
{%
\vspace{2pt}\end{adjustwidth}\endMakeFramed%
}


% |||||||||||||||||||||||||||||||||||||||||||||||||||||||||||||||||||||||||||||||||||
% |||||||||||||||||||||||||||||||||||||||||||||||||||||||||||||||||||||||||||||||||||
% |||||||||||||||||||||||||||||||||||||||||||||||||||||||||||||||||||||||||||||||||||
% |||||||||||||||||||||||||||||||||||||||||||||||||||||||||||||||||||||||||||||||||||
% |||||||||||||||||||||||||||||||||||||||||||||||||||||||||||||||||||||||||||||||||||
\begin{document}
\begin{center}
    \zihao{-2} {\heiti 一元一次方程应用题 }
\end{center}
\tableofcontents
% ||||||||||||||||||||||||||||||||||||||||||||||||||||||||||||||||||||||||||||||

% ||||||||||||||||||||||||||||||||||||||||||||||||||||||||||||||||||||||||||||||
\newpage
\section{绪论}
\subsection{做题方法}
\begin{wa}
\quad \quad (1)审题:认真审题,找出等量关系.

(2)设未知数:根据提问,巧设未知数.

(3)列方程:用未知数表示出有关含字母的式子,利用已找出的等量关系列方程.

(4)解方程:解所列的方程,求出未知数的值.

(5)检验,作答:检验所求出的未知数的值是否是方程的解,是否符合实际,检验后写出答案.(注意带上单位)
\end{wa}
% ||||||||||||||||||||||||||||||||||||||||||||||||||||||||||||||||||||||||||||||

% ||||||||||||||||||||||||||||||||||||||||||||||||||||||||||||||||||||||||||||||
\newpage
\section{行程问题}
\subsection{相遇追及}
\subsubsection*{相关概念}
\begin{wa}
\quad \quad1.行程问题中的三个基本量及其关系:

路程=速度×时间 \quad 时间=路程÷速度 \quad 速度=路程÷时间

\end{wa}
\begin{wa}
    \quad \quad    2.行程问题基本类型

(1)相遇问题:快行距+慢行距=原距

(2)追及问题:快行距-慢行距=原距
\end{wa}
\subsubsection*{题目}
1、从甲地到乙地,某人步行比乘公交车多用 3.6 小时,已知步行速度为每小时 8 千米,公交车的速度为每小时 40 千米,设甲、乙两地相距 x 千米,则列方程

~\\
~\\
~\\
~\\
~\\
2、甲、乙两人在相距 18 千米的两地同时出发,相向而行,1 小时48 分相遇,如果甲比乙早出发 40 分钟,那么在乙出发 1 小时 30 分相遇,当甲比乙每小时快 1 千米时,求甲、乙两人的速度

~\\
~\\
~\\
~\\
~\\
3、某人从家里骑自行车到学校。若每小时行 15 千米,可比预定时间早到 15 分钟;若每小时行 9 千米,可比预定时间晚到 15 分钟;求从家里到学校的路程有多少千米?

~\\
~\\
~\\
~\\
~\\
4、在 800 米跑道上有两人练习中长跑,甲每分钟跑 320 米,乙每分钟跑 280 米,两人同时同地同向起跑,t 分钟后第一次相遇,求 t5、一列客车车长 200 米,一列货车车长 280 米,在平行的轨道上相向行驶,从两车头相遇到两车车尾完全离开经过 16 秒,已知客车与货车的速度之比是 3:2,问两车每秒各行驶多少米

~\\
~\\
~\\
~\\
~\\
6、与铁路平行的一条公路上有一行人与骑自行车的人同时向南行进。行人的速度是每小时 3.6km,骑自行车的人的速度是每小时10.8km。如果一列火车从他们背后开来,它通过行人的时间是 22 秒,
通过骑自行车的人的时间是 26 秒。
⑴行人的速度为每秒多少米?
⑵这列火车的车长是多少米?

~\\
~\\
~\\
~\\
~\\
7、休息日我和妈妈从家里出发一同去外婆家,我们走了 1 小时后,爸爸发现带给外婆的礼品忘在家里,便立刻带上礼品以每小时 6千米的速度去追我们,如果我和妈妈每小时行 2 千米,从家里到外婆
家需要 1 小时 45 分钟,问爸爸能在我和妈妈到外婆家之前追上我们吗?

~\\
~\\
~\\
~\\
~\\
8、一次远足活动中,一部分人步行,另一部分乘一辆汽车,两部分人同地出发。汽车速度是 60 千米/时,步行的速度是 5 千米/时,步行者比汽车提前 1 小时出发,这辆汽车到达目的地后,再回头接步行的这部分人。出发地到目的地的距离是 60 千米。问:步行者在出发后经过多少时间与回头接他们的汽车相遇(汽车掉头的时间忽略不计)9、一列火车长 150 米,以每秒 15 米的速度通过 600 米的隧道,从火车进入隧道口算起,到这列火车完全通过隧道所需时间是

~\\
~\\
~\\
~\\
~\\
10、某人计划骑车以每小时 12 千米的速度由 A 地到 B 地,这样便可在规定的时间到达 B 地,但他因事将原计划的时间推迟了 20 分,便只好以每小时 15 千米的速度前进,结果比规定时间早 4 分钟到达 B地,求 A、B
\subsection{行船飞行}
\subsubsection*{相关概念}
\begin{wa}
\quad \quad 顺水(风)速度=静水(风)速度+水流(风)速度

逆水(风)速度=静水(风)速度-水流(风)速度

水流速度=(顺水速度-逆水速度)÷2

此类题目中不变的量是出发点到终点的距离
\end{wa}
\subsubsection*{题目}
1、 一艘船在两个码头之间航行,水流的速度是 3 千米/时,顺水航行需要 2 小时,逆水航行需要 3 小时,求两码头之间的距离。

~\\
~\\
~\\
~\\
~\\
2、一架飞机飞行在两个城市之间,风速为每小时 24 千米,顺风飞行需要 2 小时 50 分钟,逆风飞行需要 3 小时,求两城市间的距离。

~\\
~\\
~\\
~\\
~\\
3、小明在静水中划船的速度为 10 千米/时,今往返于某条河,逆水用了 9 小时,顺水用了 6 小时,求该河的水流速度

~\\
~\\
~\\
~\\
~\\
% ||||||||||||||||||||||||||||||||||||||||||||||||||||||||||||||||||||||||||||||

% ||||||||||||||||||||||||||||||||||||||||||||||||||||||||||||||||||||||||||||||
\subsection{火车行驶}
\subsubsection*{相关概念}
\begin{wa}
\quad \quad (1)完全通过隧道(桥):行驶路程=隧道(桥)长+火车长度

(2)整列火车在隧道(桥)上:行驶路程=隧道(桥)长-火车长度

此类题目的不变量是火车的速度,通常可以从这个角度作为切入点列方程
\end{wa}
\subsubsection*{题目}
1.一列火车匀速行驶经过一座桥,火车完全通过桥共用了50s。整列火车在桥上的时间为30s。已知桥长1200m,求火车的长度和速度

~\\
~\\
~\\
~\\
~\\
2.一列火车匀速行驶,经过一条长800m的隧道。从车头开始进入隧道到车尾离开隧道一共需要50s的时间。在隧道中央的顶部有一盏灯,垂直向下发光,照在火车上的时间是18s,则该火车的长度?

~\\
~\\
~\\
~\\
~\\
3.一列火车匀速行驶,完全通过一条长350m的隧道需要10s的时间。隧道顶部有一盏灯,垂直向下发光,照在火车上的时间是5s,则该火车的速度?

~\\
~\\
~\\
~\\
~\\
4.一已知某铁路桥长1600m,现有一列火车从桥上通过。测得火车开始上桥到完全通过桥共用90s。整列火车完全在桥上的时间是70s,则该火车的长度?

~\\
~\\
~\\
~\\
~\\
5.甲、乙两列火车的长分别为144m和180m,甲车比乙车每秒多行4m。
(1)两列车相向行驶,从相遇到全部错开(从两车头相遇到两车尾离开)需9s,两车的速度分别是多少?
(2)若同向行驶,甲车的车头从乙车的车尾追及到甲车全部超出乙车,需多少秒?

\subsection{环形跑道/时钟}
\subsubsection*{相关概念}
\subsubsection*{习题}
1、在 6 点和 7 点之间,什么时刻时钟的分针和时针重合?

~\\
~\\
~\\
~\\
~\\
2、甲、乙两人在 400 米长的环形跑道上跑步,甲分钟跑 240 米,乙每分钟跑 200 米,二人同时同地同向出发,几分钟后二人相遇?若背向跑,几分钟后相遇?

~\\
~\\
~\\
~\\
~\\
3、在 3 时和 4 时之间的哪个时刻,时钟的时针与分针:⑴重合;⑵成平角;⑶成直角;

~\\
~\\
~\\
~\\
~\\ 
% ||||||||||||||||||||||||||||||||||||||||||||||||||||||||||||||||||||||||||||||

% ||||||||||||||||||||||||||||||||||||||||||||||||||||||||||||||||||||||||||||||
\newpage
\section{工程问题}
\subsection{相关概念}
\begin{wa}
 \quad \quad   工作量=工作效率×工作时间=人均工作效率×人数×工作时间

    工作效率=工作量÷工作时间

\end{wa}   
通常情况,工作总量设为1,完成某工作的各个工作量之和为工作总量

在处理工程问题类的题目时,我们可以通过列表来解决,比如:

\begin{table}[htbp]
    \centering
    \begin{tabular}{|c|c|c|c|}
        \hline \ & 工作效率 & 工作时间 & 工作量 \\
        \hline A & \ & \ & \ \\
        \hline B & \ & \ & \ \\
        \hline
    \end{tabular}
\end{table}
\begin{center}
    A的工作量+B的工作量=1
\end{center}
\subsection{习题}
1、一项工程,甲单独做要 10 天完成,乙单独做要 15 天完成,两人合做 4 天后,剩下的部分由乙单独做,还需要几天完成?

~\\
~\\
~\\
~\\
~\\
2、甲、乙两个工程队合做一项工程,乙队单独做一天后,由甲、乙两队合做两天后就完成了全部工程.已知甲队单独做所需天数是乙队单独做所需天数的 ,问甲、乙两队单独做,各需多少天?

~\\
~\\
~\\
~\\
~\\
3、某工程,甲单独完成续 20 天,乙单独完成续 12 天,甲乙合干6 天后,再由乙继续完成,乙再做几天可以完成全部工程?

~\\
~\\
~\\
~\\
~\\
4、某工作,甲单独干需用 15 小时完成,乙单独干需用 12 小时完成, 若甲先干 1 小时、乙又单独干 4 小时,剩下的工作两人合作,问:再用几小时可全部完成任务?

~\\
~\\
~\\
~\\
~\\
5、一水池,单开进水管 3 小时可将水池注满,单开出水管 4 小时可将满池水放完。现对空水池先打开进水管 2 小时,然后打开出水管,使进水管、出水管一起开放,问再过几小时可将水池注满?

~\\
~\\
~\\
~\\
~\\
6、一水池有一个进水管,4 小时可以注满空池,池底有一个出水管,6 小时可以放完满池的水.如果两水管同时打开,那么经过几小时可把空水池灌满?

~\\
~\\
~\\
~\\
~\\
7、一项工程 300 人共做, 需要 40 天,如果要求提前 10 天完成,问需要增多少人?

~\\
~\\
~\\
~\\
~\\
8、整理一批图书,由一个人做要 40 小时完成。现计划由一部分人先做 4 小时,再增加 2 人和他们一起做 8 小时,完成这项工作。假设这些人的工作效率相同,具体先安排多少人工作。

~\\
~\\
~\\
~\\
~\\
% ||||||||||||||||||||||||||||||||||||||||||||||||||||||||||||||||||||||||||||||

% ||||||||||||||||||||||||||||||||||||||||||||||||||||||||||||||||||||||||||||||
\newpage
\section{配套问题}
\subsection{解题思路}
\begin{wa}
    \quad \quad 在求解配套问题时,根据比例找等量关系,列表表示各部分的值,代入最开始的等式即可. 
\end{wa}
比如:某车间有22名工人,每人每天可生产1200个螺钉或2000个螺母. 1个螺钉需要配2个螺母,为使每天生产的螺钉和螺母刚好配套,应安排生产螺钉和螺母的工人各多少名?

分析:

螺钉数量:螺母数量=1:2 \ $\rightarrow$ \ \textcolor{red}{"螺母数量=2×螺钉数量"}

设生产螺钉的人数为x人,则生产螺母的人数为(22-x)人,于是可以得出以下表格
\begin{table}[htbp]
    \centering
    \begin{tabular}{|l|c|c|c|}
        \hline \ & 人数 & 数量  \\
        \hline 螺钉 & x & 1200x \\
        \hline 螺母 & 22-x & 2000(22-x) \\
        \hline
    \end{tabular}
\end{table}
\subsection{习题}
1.某车间有 28 名工人生产螺栓和螺母,每人每小时平均能生产螺栓 12 个或螺母 18 个,应如何分配生产螺栓和螺母的工人,才能使螺栓和螺母正好配套(一个螺栓配两个螺母)?

~\\
~\\
~\\
~\\
~\\
2.机械厂加工车间有 85 名工人,平均每人每天加工大齿轮 16 个或小齿轮 10 个,已知 2 个大齿轮与 3 个小齿轮配成一套,问需分别安排多少名工人加工大、小齿轮,才能使每天加工的大小齿轮刚好配套?

~\\
~\\
~\\
~\\
~\\
3.某部队派出一支有 25 人组织的小分队参加防汛抗洪斗争,若每人每小时可装泥土 18 袋或每 2 人每小时可抬泥土 14 袋,如何安排好人力,才能使装泥和抬泥密切配合,而正好清场干净。

~\\
~\\
~\\
~\\
~\\
4.某车间加工机轴和轴承,一个工人每天平均可加工 15 个机轴或10 个轴承。该车间共有 80 人,一根机轴和两个轴承配成一套,问应分配多少个工人加工机轴或轴承,才能使每天生产的机轴和轴承正好配套。

~\\
~\\
~\\
~\\
~\\
5.某厂生产一批西装,每 2 米布可以裁上衣 3 件,或裁裤子 4条,现有花呢 240 米,为了使上衣和裤子配套,裁上衣和裤子应该各用花呢多少米?

~\\
~\\
~\\
~\\
~\\
% ||||||||||||||||||||||||||||||||||||||||||||||||||||||||||||||||||||||||||||||

% ||||||||||||||||||||||||||||||||||||||||||||||||||||||||||||||||||||||||||||||
\newpage
\section{利润问题}
\subsection{相关概念}
\begin{wa}
    \quad \quad 进价进货价商品的成本价

    标价商品出售时所标明的价格

    售价商品在出售时的实际价格
\end{wa}
\begin{wa}
    \quad \quad 利润=售价-进价=进价×利润率

    售价=进价+利润=进价×(1+利润率)=标价×$\dfrac{\text{折扣数}}{10}$

    利润率=$\dfrac{\text{利润}}{\text{进价}}$×100\%=$\dfrac{\text{售价-进价}}{\text{进价}}$×100\%
\end{wa}
\subsection{习题}
1、一商场把彩电按标价的九折出售,仍可获利 20\%,如果该彩电的进货价是 2400 元,那么彩电的标价是多少元?

~\\
~\\
~\\
~\\
~\\
2、 一家服装店将某种服装按成本提高 40\%后标价,又以八折优惠卖出,结果每件仍获利 15 元,求这种服装每件的成本.

~\\
~\\
~\\
~\\
~\\
3、 某商品的销售价格每件 900 元,为了参加市场竞争,商店按售价的九折再让利 40 元销售,些时仍可获利 10\%,求此商品的进价.

~\\
~\\
~\\
~\\
~\\
4. 商店里有种型号的电视机,每台售价 1200 元,可盈利 20\%,现有一客商以 11500 元的总价购买了若干台这咱型号的电视机,这样商店仍有 15\%的利润,问客商买了几台电视机?

~\\
~\\
~\\
~\\
~\\
5.某商店在某一时间以每件 60 元的价格卖出两件衣服,其中一件盈利 25\%,另一件亏损 25\%,卖这两件衣服总的是盈利还是亏损,或是不盈不亏?

~\\
~\\
~\\
~\\
~\\
6.某商店开张为吸引顾客,所有商品一律按八折优惠出售,已知某种旅游鞋每双进价为 60 元,八折出售后,商家所获利润率为 40\%。问这种鞋的标价是多少元?优惠价是多少?

~\\
~\\
~\\
~\\
~\\
7.工艺商场按标价销售某种工艺品时,每件可获利 45 元;按标价的八五折销售该工艺品 8 件与将标价降低 35 元销售该工艺品 12 件所获利润相等.该工艺品每件的进价、标价分别是多少元?

~\\
~\\
~\\
~\\
~\\
8.某高校共有 5 个大餐厅和 2 个小餐厅.经过测试:同时开放 1个大餐厅、2 个小餐厅,可供 1680 名学生就餐;同时开放 2 个大餐厅、1 个小餐厅,可供 2280 名学生就餐.
(1)求 1 个大餐厅、1 个小餐厅分别可供多少名学生就餐;
(2)若 7 个餐厅同时开放,能否供全校的 5300 名学生就餐?

~\\
~\\
~\\
~\\
~\\
9.(2006·益阳市)八年级三班在召开期末总结表彰会前,班主任安排班长李小波去商店买奖品,下面是李小波与售货员的对话:

李小波:阿姨,您好!

售货员:同学,你好,想买点什么?

李小波:我只有 100 元,请帮我安排买 10 支钢笔和 15 本笔记本. 

售货员:好,每支钢笔比每本笔记本贵 2 元,退你 5 元,请清点好,再见. 

根据这段对话,你能算出钢笔和笔记本的单价各是多少吗?

~\\
~\\
~\\
~\\
~\\
10.某地区居民生活用电基本价格为每千瓦时 0.40 元,若每月用电量超过 a 千瓦则超过部分按基本电价的 70\%收费.

(1)某户八月份用电 84 千瓦时,共交电费 30.72 元,求 a.

(2)若该用户九月份的平均电费为 0.36 元,则九月份共用电多少千瓦? 应交电费是多少元

% ||||||||||||||||||||||||||||||||||||||||||||||||||||||||||||||||||||||||||||||

% ||||||||||||||||||||||||||||||||||||||||||||||||||||||||||||||||||||||||||||||
\newpage
\section{计费问题}
\subsection{解题思路}
\subsection{习题}
% ||||||||||||||||||||||||||||||||||||||||||||||||||||||||||||||||||||||||||||||

% ||||||||||||||||||||||||||||||||||||||||||||||||||||||||||||||||||||||||||||||
\newpage
\section{其他补充}
\subsection{数字问题}
\subsubsection*{相关概念}
\begin{wa}

\quad \quad   1.数的表示方法:一个三位数,一般可设百位数字为 a,十位数字是 b,个位数字为 c(其中a、b、c 均为整数,且 1≤a≤9,0≤b≤9, 0≤c≤9),则这个三位数表示为:100a+10b+c

2.多个数字的表示方法:两个连续整数,较大的数比较小的大 1;偶数用 2n 表示,连续的偶数用 2n+2 或 2n-2 表示;奇数用2n+1或2n—1表示
\end{wa}

\subsubsection*{习题}
1、一个两位数,个位数字比十位数字小 1,这个两位数的个位十位互换后,它们的和是 33,求这个两位数.

~\\
~\\
~\\
~\\
~\\
2.一个两位数,十位上的数字与个位上的数字之和为 11,如果把十位上的数字与个位上的数字对调,那么得到的新数就比原数大 63,求原来的两位数.

~\\
~\\
~\\
~\\
~\\
3.三位数的数字之和是 17,百位上的数字与十位上的数字的和比个位上的数大 3,如把百位上的数字与个位上的数字对调,所得的新数比原数大 495,求原数.

~\\
~\\
~\\
~\\
~\\
4.有一个两位数,它的十位上的数字比个位上的数字小 3,十位上的数字与个位上的数字之和等于这个两位数的 ,求这个两位数.

~\\
~\\
~\\
~\\
~\\
5.有一个三位数,个位数字为百位数字的 2 倍,十位数字比百位数字大 1,若将此数个位与百位顺序对调(个位变百位)所得的新数比原数的 2 倍少 49,求原数.

~\\
~\\
~\\
~\\
~\\
\subsection{年龄问题}
\subsubsection*{相关概念}
\subsubsection*{习题}
1.某同学今年 15 岁,他爸爸今年 39 岁,问几年以后,爸爸的年龄是这位同学年龄的 2 倍?

~\\
~\\
~\\
~\\
~\\
2.三位同学甲乙丙,甲比乙大 1 岁,乙比丙大 2 岁,三人的年龄之和为 41,求乙同学的年龄.

~\\
~\\
~\\
~\\
~\\
3.兄弟二人今年分别为 15 岁和 9 岁,多少年后兄的年龄是弟的年龄的 2 倍?

~\\
~\\
~\\
~\\
~\\
4.今年哥俩的岁数加起来是 55 岁。曾经有一年,哥哥的岁数与今年弟弟的岁数相同,那时哥哥的岁数恰好是弟弟岁数的两倍.哥哥今年几岁?

~\\
~\\
~\\
~\\
~\\

\subsection{比赛积分}
\subsubsection*{相关概念}
\subsubsection*{习题}
1.某企业对应聘人员进行英语考试,试题由 50 道选择题组成,评分标准规定:每道题的答案选对得 3 分,不选得 0 分,选错倒扣 1分。已知某人有 5 道题未作,得了 103 分,则这个人选错了几道题?

~\\
~\\
~\\
~\\
~\\
2.丰台二中进行小测(数学),一共 10 道题。每做对一道得 8分,错一道扣 5 分。一位同学得了 41 分。问那位同学对几道,错几道?

~\\
~\\
~\\
~\\
~\\
3.一份试卷共有 25 道题,每道题都给出了 4 个答案,其中只有一个正确答案,每道题选对得 4 分,不选或错选倒扣 1 分,如果一个学生得 90 分,那么他做对了多少道题。

~\\
~\\
~\\
~\\
~\\
\subsection{比例问题}
\subsubsection*{相关概念}
\subsubsection*{习题}
1、学校有电视和幻灯机共 90 台,已知电视机和幻灯机的台数比为 2:3,求学校有电视机和幻灯机各多少台?

~\\
~\\
~\\
~\\
~\\
2. 如果两个课外兴趣小组共有人数 54 人,两个小数的人数之比是 4:5;如果设人数少的一组有 4x 人,那么人数多的一组有多少人?

~\\
~\\
~\\
~\\
~\\
3. 甲乙两人身上的钱数之比为 7:6,两人去商店买东西后,甲花去 50 元,乙花去 60 时,此时他们身上的钱数之比为 3:2,则他们身上余下的钱数分别是多少?

~\\
~\\
~\\
~\\
~\\
\newpage
\subsection{浓度问题}
\subsubsection*{相关概念}
\begin{wa}
   \quad \quad 溶质质量+溶剂质量=溶液质量

    浓度=$\dfrac{\text{溶质重量}}{\text{溶液重量}}$×100\%

    溶液重量×浓度=溶质重量

    溶质重量÷浓度=溶液重量
\end{wa}
\subsubsection*{习题}
\noindent 1.\quad (1)有含盐20\%的盐水5千克,要配制成含盐8\%的盐水,需加水多少千克?

(2)某化工厂现有浓度为15\%的稀硫酸175千克,要把它配成浓度为25\%的硫酸,需要加入浓度为50\%的硫酸多少千克?

~\\
~\\
~\\
~\\
~\\
2. 今需将浓度为80\%和15\%的两种农药配制成浓度为20\%的农药4千克,问两种农药应各取多少千克?

~\\
~\\
~\\
~\\
~\\
3. 甲、乙两块合金,含银和铜的比分别是甲为4:3,乙为7:9,今从两块合金中各取多少千克,能得到含银84千克、含铜82千克的新合金?

~\\
~\\
~\\
~\\
~\\
4.有甲、乙两种铜和银的合金,甲种合金含银25\%,乙种合金含银537.5\%,现在要熔制含银30\%的合金 100 千克,两种合金应各取多少千克?

~\\
~\\
~\\
~\\
~\\
\subsection{方案选择}
\subsubsection*{相关概念}
\subsubsection*{习题}
1.某蔬菜公司的一种绿色蔬菜,若在市场上直接销售,每吨利润为1000元,经粗加工后销售,每吨利润可达4500元,经精加工后销售,每吨利润涨至7500元,当地一家公司收购这种蔬菜140吨,该公司的加工生产能力是:如果对蔬菜进行精加工,每天可加工16吨,如果进行精加工,每天可加工6吨, 但两种加工方式不能同时进行,受季度等条件限制,公司必须在15天将这批蔬菜全部销售或加工完毕,为此公司研制了三种可行方案:

方案一:将蔬菜全部进行粗加工.

方案二:尽可能多地对蔬菜进行粗加工,没来得及进行加工的蔬菜, 在市场上直接销售.

方案三:将部分蔬菜进行精加工,其余蔬菜进行粗加工,并恰好15天完成.

你认为哪种方案获利最多?为什么?

~\\
~\\
~\\
~\\
~\\
~\\
~\\
~\\
~\\
~\\
2.某家电商场计划用9万元从生产厂家购进50台电视机.已知该厂家生产3种不同型号的电视机,出厂价分别为A种每台1500元,B种每台2100元,C种每台2500元.

(1)若家电商场同时购进两种不同型号的电视机共50台,用去9万元,请你研究一下商场的进货方案.

(2)若商场销售一台A种电视机可获利150元,销售一台B种电视机可获利200元, 销售一台C种电视机可获利250元,在同时购进两种不同型号的电视机方案中,为了使销售时获利最多,你选择哪种方案?

\end{document}