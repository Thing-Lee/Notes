\documentclass[oneside]{book} %oneside消除奇偶页出现的空页问题,页眉也会随之改变
\usepackage[zihao=-4]{ctex} %引入中文宏包
\usepackage{ulem}%提供\uline命令引入高低一致的下划线
\usepackage{amsmath}%数学公式排版
\usepackage{amssymb}%用命令 \mathbb
\usepackage{geometry}%编辑页面边距
\usepackage{pifont}%编辑带圈数字
\usepackage{fancyhdr}%风格,页眉页脚
\usepackage[backref]{hyperref}%超链接
\usepackage{color}%字体颜色
\usepackage{extarrows}%输入带文字的长等号
\usepackage[dvipsnames,svgnames]{xcolor}
\usepackage[strict]{changepage} % 提供一个 adjustwidth 环境
\usepackage{framed} % 实现方框效果
\usepackage{amsthm}

\usepackage{tikz}
\usetikzlibrary{mindmap,trees}
\usepackage{verbatim}


\hypersetup{colorlinks=true,linkcolor=cyan,anchorcolor=megenta,citecolor=yellow}
\geometry{right=2.5cm,left=2.5cm,top=2.5cm,bottom=2.5cm}
\renewcommand{\headrulewidth}{0pt}%去掉页眉分划
\pagestyle{fancy}
\fancyhead[L]{}
\fancyhead[R]{\rightmark}

% ------------------******-------------------
\newtheorem{lemma}{推论}[section]  %推论
% environment derived from framed.sty: see leftbar environment definition
\definecolor{lashade}{rgb}{0.97,0.96,0.99} % 方框底部颜色
% ------------------******-------------------
% 注意行末需要把空格注释掉,不然画出来的方框会有空白竖线
\newenvironment{la}{%
\def\FrameCommand{%
\hspace{1pt}%
{\color{RoyalPurple}\vrule width 2pt}%竖线颜色
{\color{lashade}\vrule width 4pt}%
\colorbox{lashade}%
}%
\MakeFramed{\advance\hsize-\width\FrameRestore}%
\noindent\hspace{-4.55pt}% disable indenting first paragraph
\begin{adjustwidth}{}{7pt}%
\vspace{2pt}\vspace{2pt}%
\normalfont %环境字体设置
}
{%
\vspace{2pt}\end{adjustwidth}\endMakeFramed%
}

\newtheorem*{example}{例}
\newtheorem{examp}{例}
\definecolor{egshade}{rgb}{0.96,0.96,0.96}% 
\newenvironment{eg}{%
\def\FrameCommand{%
\hspace{1pt}%
{\color{Gray}\vrule width 2pt}%竖线颜色
{\color{egshade}\vrule width 4pt}%
\colorbox{egshade}%
}%
\MakeFramed{\advance\hsize-\width\FrameRestore}%
\noindent\hspace{-4.55pt}% 
\begin{adjustwidth}{}{7pt}%
\vspace{2pt}\vspace{2pt}%
\normalfont %环境字体设置
}
{%
\vspace{2pt}\end{adjustwidth}\endMakeFramed%
}

\newtheorem{definition}{定义}[section]  
\newtheorem{defini}{}
\newtheorem{theorem}{定理}[section]  
\newtheorem{theo}{}  
\definecolor{tdshade}{rgb}{0.94,0.97,0.93} % 
\newenvironment{td}{%
\def\FrameCommand{%
\hspace{1pt}%
{\color{Green}\vrule width 2pt}%竖线颜色
{\color{tdshade}\vrule width 4pt}%
\colorbox{tdshade}%
}%
\MakeFramed{\advance\hsize-\width\FrameRestore}%
\noindent\hspace{-4.55pt}% 
\begin{adjustwidth}{}{7pt}%
\vspace{2pt}\vspace{2pt}%
\normalfont %
}
{%
\vspace{2pt}\end{adjustwidth}\endMakeFramed%
}


\newtheorem{warning}{注}[section]  
\definecolor{washade}{rgb}{0.99,0.95,0.94} % 方框底部颜色
\newenvironment{wa}{%
\def\FrameCommand{%
\hspace{1pt}%
{\color{LightCoral}\vrule width 2pt}%竖线颜色
{\color{washade}\vrule width 4pt}%
\colorbox{washade}%
}%
\MakeFramed{\advance\hsize-\width\FrameRestore}%
\noindent\hspace{-4.55pt}% disable indenting first paragraph
\begin{adjustwidth}{}{7pt}%
\vspace{2pt}\vspace{2pt}%
\normalfont %
}
{%
\vspace{2pt}\end{adjustwidth}\endMakeFramed%
}





% ------------------正文-------------------

\begin{document} 

\tableofcontents %生成目录
% --------------第一章----------------
\chapter{第一章 \ 绪论}
% ---------------1.1----------------        
\section{微分方程概述}
\begin{center} 
    \begin{tikzpicture}
        \path[small mindmap,concept color = red!65,text = black]
            node[concept] {方程}
            [clockwise from=25]
            child[concept color = red!50] 
            {
                node[concept] {未知量(数)}
                [clockwise from=25]
                child [concept color = red!40] 
                { 
                    node[concept] {代数方程} 
                }
                child [concept color = red!40] 
                { 
                    node[concept] {超越方程} 
                }     
            }
            child[concept color = red!50] 
            {
                node[concept] {未知函数}
                [clockwise from=0]
                child [concept color = red!40] 
                { 
                    node[concept] {函数方程} 
                    child [concept color = red!30] 
                    { 
                        node[concept] {微分方程} 
                    }
                }
            };
    \end{tikzpicture}
\end{center}


\indent \textbf{方程:}含有未知量(数)的等式或关系式

\textbf{代数方程:}含有超越数。特点:方程的解为有限个或无限个实数

\textbf{函数方程:}其未知量为函数。特点:方程的解为有限个或无限个函数

\textbf{\textcolor{red}{微分方程:}}自变量、未知函数及函数的导数(或微分)组成的关系式

\textbf{常微分方程:}自变量只有一个时的微分方程

\textbf{偏微分方程:}自变量有多个时的微分方程


% ----------------1.2-----------------
\section{基本概念}
\subsubsection{1.阶数}
\noindent \pmb{定义:}在一个微分方程中所出现的未知函数的导数的最高阶数n称为该方程的\textbf{\textcolor{red}{阶}}

一阶微分方程:$n = 1$

高阶微分方程:$n > 1$

\noindent \pmb{基本表示:}

\noindent 一阶常微分方程的一般隐式形式
$$ F(x,y,y') = 0$$
一阶常微分方程的一般显式形式
$$y' = f(x,y)$$
$n$阶隐方程的一般形式
$$ F(x,y,y',\dots,y^{(n)}) = 0$$
$n$阶显方程的一般形式
$$y^{(n)} = f(x,y,y',\dots,y^{(n-1)})$$


\subsubsection{2.线性、非线性}
\noindent \pmb{定义:}若方程的左端为$y$,$\dfrac{dy}{dx}$,$\dots$,$\dfrac{d^n y}{d x^n} $,的一次有理整式,则称它为
线性微分方程,否则,称它为非线性微分方程

\noindent \pmb{一般形式:}
$$a_{0}(x)y^{(n)} + a_{1}(x)y^{(n - 1)} + \dots + a_{n}(x)y = g(x)$$
$$\frac{d^n y}{d x^n} + a_{1}(x)\frac{d^{n-1} y}{d x^{n-1}} + \dots + a_{n -1}(x)\frac{dy}{dx} + a_{n}(x)y = f(x)  $$


\subsubsection{3.解和隐式解}
\noindent \pmb{定义:}若将函数$y = \varphi(x)$代入方程后使微分方程有意义且为恒等式,为方程的\textbf{解};若方程的解是某关系式的隐函数,称这个关系式为该方程
的\textbf{隐式解}。把方程解和隐式解统称为方程的解。

\subsubsection{4.通解和特解}
\noindent \pmb{定义:}$n$阶微分方程的含有$n$个独立的任意常数$c_{1},c_{2},\dots,c_{n}$的解$y = \varphi(x,c_{1},c_{2},\dots,c_{n})$称为$n$阶方程的\textcolor{red}{\textbf{通解}};常微分方程满足某个初始条件的解称为微分方程的特解

\noindent \pmb{定解条件:}确定微分方程的一个特定的解,给出这个解所必需的条件

\noindent \pmb{初值条件:}对于$n$阶方程$y^{(n)} = f(x,y,y',\dots,y^{(n-1)})$的初值条件可表示为
$$y(x_{0}) = y_{0} \quad y'(x_{0}) = y'_{0} \quad y''(x_{0}) = y''_{0} \quad \dots \quad y^{(n-1)}(x_{0}) = y^{(n-1)}_{0}$$
\pmb{定解问题:}求微分方程满足定解条件的解

\noindent \pmb{初值问题:}当定解问题为初值条件时相应的定解问题
n阶方程初值问题的表示
\[
\begin{cases}
    y^{(n)} = f(x,y,y',\dots,y^{(n-1)}) \\
    y(x_{0}) = y_{0} , y'(x_{0}) = y'_{0} , y''(x_{0}) = y''_{0} , \dots , y^{(n-1)}(x_{0}) = y^{(n-1)}_{0}
\end{cases}\]
一阶和二阶方程初值问题的表示
\[
\begin{cases}
    y' = f(x,y) \\
    y(x_{0}) = y_{0}
\end{cases}
\begin{cases}
    y'' = f(x,y,y') \\
    y(x_{0}) = y_{0} , y'(x_{0}) = y'_{0}
\end{cases}
\]
\pmb{特解:}满足初值条件的解

\subsubsection{5.积分曲线和向量场}
\noindent \pmb{积分曲线:}一阶微分方程$\dfrac{dy}{dx}$的解$y = \varphi(x)$表示$Oxy$平面上的一条曲线;通解$y = \varphi(x,c)$表示平面上的一族曲线,称为\textbf{积分曲线族};特解$\varphi(x_{0}) = y_{0}$则为过点$(x_{0},y_{0})$的一条积分曲线

\noindent \pmb{方向场(向量场):}$f(x , y) = \dfrac{dy}{dx}$在$Oxy$平面某区域$D$上定义过各点的小线段的斜率方向,这样的区域$D$称为向量场

\noindent \pmb{常微分方程求解的几何意义:}在方向场中寻求一条曲线,使这条曲线上每一点切线的方向等于方向场中该点的方向

\subsubsection{6.计算举例}
已知$S = S(t)$$S' = S'(t)$
\begin{align*}
    \dfrac{d^{2}s}{d t^{2}} & = g \quad \quad \quad \quad \quad \quad \Rightarrow 
    \dfrac{d^{2}s}{d t^{2}} = \dfrac{dS'}{dt} = g \\
    dS' & = g dt \quad \quad \quad \quad \quad \Rightarrow 
    \int dS' = \int g dt \quad \quad \quad \quad \quad \Rightarrow 
    S' = gt + C_{1} \quad \quad \Rightarrow
    \dfrac{dS}{dt} = gt + C_{1} \\
    dS & = (gt + C_{1}) dt \quad \ \Rightarrow 
    \int dS = \int (gt + C_{1}) dt \quad \quad \Rightarrow 
    S = \dfrac{1}{2} gt^{2} + C_{1} + C_{2}
\end{align*}


% -------------------第二章-------------------
\chapter{第二章 \ 一阶微分方程的初等解法}
% ------------------2.1---------------------
\section{变量分离方程与变量变换}
\subsection*{变量分离方程}

\noindent \textbf{定义:}形如$\dfrac{dy}{dx} = f(x) \varphi(y)$的方程,$f(x)$,$\varphi(y)$分别是$x$,$y$的连续函数

\noindent \textbf{求解步骤:}

\ding{172}若$\varphi(y) \neq 0$,则有
$$\frac{dy}{\varphi(y)} = f(x)dx$$
\indent 两边同时积分得
$$\int \frac{dy}{\varphi(y)} = \int f(x) \,dx + c \ \footnote{在ODE中此式代指一个函数,所以等式后加常数c;不同于数分中代指的是一族函数} $$
\indent \ding{173}若$\varphi(y_{0}) = 0$,则有特解
$$y = y_{0}$$
综上:\dots

\textbf{注:}变量分离方程可能是线性($\dfrac{dy}{dx} = xy$),也可能是非线性($\dfrac{dy}{dx} = xy^{2}$)
\\ \hspace*{\fill} \\
\noindent \pmb{例2:}求以下方程通解
$$\frac{dy}{dx} =P(x)y$$
\pmb{解:}
\ding{172}若$\varphi(y) \neq 0$,则有
$$\frac{dy}{y} = P(x)dx$$
\indent 两边同时积分得
$$\ln \left\lvert y\right\rvert = \int P(x) \,dx + \overline{c} $$
\indent 通解为
$$\left\lvert y \right\rvert = e^{\int \limits P(x) \,dx + \overline{c}}$$
\indent 即
$$y = \pm e^{\overline{c}} \cdot e^{\int \limits P(x) \,dx } = ce^{\int \limits P(x) \,dx } \quad  (c = \pm e^{\overline{c} })$$
\indent \ding{173}若$\varphi(y_{0}) = 0$,则上式$c = 0$

\indent 故方程的解为\textcolor{red}{$y = ce^{\int \limits P(x) \,dx }$},其中$c$为任意常数

\subsection*{可化为变量分离方程的类型}
\noindent \pmb{齐次微分方程:}

\noindent \textbf{定义:}方程形如$$\dfrac{dy}{dx} = g\left(\dfrac{y}{x}\right) $$

\noindent \textbf{求解步骤:}

作变量变换
$$u = \frac{y}{x}$$
\indent 两边同时求导得:
$$\frac{dy}{dx} = x \frac{du}{dx} + u$$
\indent 代入上式有
$$\frac{du}{dx} = \frac{g(u) - u}{x}$$
\indent 下用变量分离的方法求解

\textbf{注:}齐次微分方程可能是线性($\dfrac{dy}{dx} = \dfrac{y}{x}$),也可能是非线性($\dfrac{dy}{dx} = \dfrac{y^{2}}{x^{2}}$)
\\ \hspace*{\fill} \\
\pmb{例4:}解方程
$$\frac{dy}{dx} = \frac{y}{x} + \tan \dfrac{y}{x}$$
\pmb{解:}
作变量变换
$$u = \frac{y}{x}$$
\indent 两边同时求导得
$$\frac{dy}{dx} = x \frac{du}{dx} + u$$
\indent 代入题给方程得
$$x \frac{du}{dx} = \tan u$$
\indent 若$\tan u \neq 0$,分离变量有
$$\cot u du = \frac{dx}{x}$$
$$\frac{d \sin u}{\sin u} = \frac{dx}{x}$$
\indent 两边同时求导得:
$$\ln \left\lvert \sin u\right\rvert = \ln \left\lvert x\right\rvert + \overline{c} $$
\indent 即有
$$\sin u = \pm e^{\overline{c}} \cdot x$$
\indent 令$\pm e^{\overline{c}} = c$,即有
$$\sin u = cx$$
\indent 若$\tan u = 0$,当$c = 0$时满足,为方程特解

综上,原方程的通解为$\sin \dfrac{y}{x} = cx$









\end{document}