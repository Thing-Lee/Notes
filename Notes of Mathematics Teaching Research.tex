\documentclass{article}
\usepackage[UTF8]{ctex}
\usepackage{amsmath}
\usepackage{amssymb}
\usepackage{geometry}%编辑页面边距
\usepackage[dvipsnames,svgnames]{xcolor}
\usepackage[strict]{changepage} % 提供一个 adjustwidth 环境
\usepackage{framed} % 实现方框效果
\usepackage{ctex}
\usepackage[backref]{hyperref}%目录链接
\usepackage{pifont}%编辑带圈数字
\usepackage{fancyhdr}%添加页眉
\pagestyle{fancy}%页眉
\fancyhead[R]{Edit By LiShiying}%页眉
\hypersetup{colorlinks=true,linkcolor=black,anchorcolor=megenta,citecolor=yellow}
\geometry{right=2.5cm,left=2.5cm,top=2.5cm,bottom=2.5cm}

% ####################高亮设置####################
\newtheorem{warning}{}  
\definecolor{washade}{rgb}{0.99,0.95,0.94} % 方框底部颜色
\newenvironment{wa}{%
\def\FrameCommand{%
\hspace{1pt}%
{\color{LightCoral}\vrule width 2pt}%竖线颜色
{\color{washade}\vrule width 4pt}%
\colorbox{washade}%
}%
\MakeFramed{\advance\hsize-\width\FrameRestore}%
\noindent\hspace{-4.55pt}% disable indenting first paragraph
\begin{adjustwidth}{}{7pt}%
\vspace{2pt}\vspace{2pt}%
\normalfont %
}
{%
\vspace{2pt}\end{adjustwidth}\endMakeFramed%
}

% ####################TikZ设置####################
\usepackage{tikz}
\usepackage{pgfplots}
\pgfplotsset{compat=1.15}
\usepackage{mathrsfs}
\usetikzlibrary{arrows}


\begin{document}
\tableofcontents
\newpage
\section{初中数学教学模式探究}

在一定的教学理论指导或教学思想下,对不同类型的教学活动要建设出多种基本框架或结构。

\subsection{教学模式概述}
\subsubsection{教学模式的特征}
1.可操作性

教学模式虽为一种教学理论,但它是具体的、可操作的。它对教学理论或教学方式进行简化,集中表现关键部分,脱离抽象理论的束缚,呈现出具体的教学操作模式和教学行为,为教师的课堂教学提供理论依据的同时也有实践操作价值,使得教师能够更好地掌控课堂。

2.指向性

无论哪一种教学模式都是在教学目标的基础上制定的,并且有不同的教学模式,其使用条件也有差异。因此,教学活动没有固定的教学模式,即所谓“教无定法”。如果运用一种教学模式有效实现了教学目标,那么该种教学模式就是适合教学活动的模式。教学模式具有指向性,因此根据不同的教学活动的特点和功能来选择适宜的教学模式。

3.稳定性

教学模式是从理论层面对教学实践活动具有的一般规律进行总结的一种稳定的方式方法。正常情况下,其与教学内容不存在直接联系,只是提供一种教学步骤,供教学时参考。教学模式作为一种教学理论,其产生与社会密切相关。因此,教学模式具有相对稳定性,它的提出取决于社会整体的教育方针和教育目标,反映的是社会历史种政治、经济、科学、文化和教育等的发展情况。

4.完整性

教学模式既是教学现实情况的体现,又是教学理论的概括,集二者于一身,具有完整的结构。从理论角度规定了其在实际教学过程中的使用要求,做到了理论于实践的结合。

5.灵活性

教学模式不是专门为特定教学活动或内容而制定的,只是一种教学理论或思想的体现,运用于实际教学过程中时要具体分析、主动适应,结合学科特点、教学内容以及教学条件和环境、师生特点等进行选择和调整。

\subsubsection{教学模式的发展与类型}

\textbf{1.经典性教学模式研究}

经典性教学模式以传统教学模式为基础,在合理继承其基本原理和理论的同时,又不断研发传授系统知识与技能的新方法。最典型的代表为范例教学模式、掌握学习教学模式和巴特勒的七阶段教学模式。

(1)范例教学模式

列举日常生活中足以体现教学本质的基础和根本因素以及具有教学意义的范例,让学生从中获取、掌握其他方面的知识,具备举一反三的能力。范例教学模式共包括四个阶段:举例说明个别事物——举例说明一类事物——举例说明具有的规律和所属范畴——举例说明扩展知识的范围

(2)掌握学习教学模式

掌握学习教学模式的出现共有三个阶段。

首先是定向。在开始一个单元的教学前,要先给学生讲授清楚这一单元的学习目标,激发学生的兴趣,促使其树立自信心,培养正确的学习方法。其次是实施,一般使用以班级为单位的集体教学法来进行单元教学。最后是检验。完成每一单元或所有教材内容的教学之后,对学生进行摸底测验,了解其掌握知识的程度,从中发现问题并及时解决。

(3)七阶段教学模式

\ding{172}情境,为学习知识技能创设的条件

\ding{173}动机,诱导或激发知识与技能学习的动力

\ding{174}组织,掌握知识的结构以及所学技能的属性

\ding{175}应用,努力在实践中运用所学的知识和技能

\ding{176}评价,对应用进行评价

\ding{177}重复,巩固温习所学知识与技能

\ding{178}推广,学习迁移,学会在新的情境应用所学的知识与技能

由以上可以看出,七阶段教学模式是对传统教学模式的一次巨大延伸和拓展,在结合传统教学理论和现代心理学相关知识的基础上提出了新的教学方法。


\textbf{2.探索性教学模式研究}

探索性教学模式是在新知识不断增加,学习越来越注重能力的背景下产生的。提倡这种教学模式的人认为,应该抛弃传统教育只注重传授知识而忽视能力培养的教育理念,应优先培养学生的探索学习能力。学生在教学过程中不应是被动的接收者,而应是主动探索者。

其模式的具体步骤是:确定问题——提出假设——验证假设——得出结论。之后模式变为:探究——研讨。

\textbf{3.程序性教学模式研究}

程序性教学模式的主要表现是在教学过程中严格遵守某种程序或算法,教学过程分为很多小的环节,各环节之间具有紧密的逻辑联系,这样就能克服传统教学模式的缺陷。这种模式的实施一般需要借助教学机器,有时甚至需要全部依赖教学机器。

\textbf{4.开发性教学模式研究}

这是一种全新的教学模式,主要目的是运用某种方法激发人的学习潜力,让人在轻松快乐的氛围中学习。其典型代表是于1995年提出的暗示教学模式。它是在外部条件作用下,对学习主体产生潜移默化的影响,使其完成学习任务,实现学习目标。

其操作过程如下:

\ding{172}创造轻松活泼的学习氛围,通过对话和游戏等环节达到复习知识的目的

\ding{173}用对话的方式引入学习内容

\ding{174}让学生背靠椅背,调整呼吸,放松自己,抓住学生的无意注意,即利用所谓瑜伽原理保持最佳的学习状态

\ding{175}在教授新内容时,教师需要借助一些形象化的手段,如声情并茂的朗诵、舒缓的音乐以及具有暗示意义的对话以及游戏等,让学生自觉地获得新知识

\ding{176}用轻松幽默的语言吸引学生的注意,完成教学

\subsubsection{教学模式的发展趋势}

\textbf{1.重视学生的趋势}
\textbf{2.重视能力的趋势}
\textbf{3.心理学的趋势}

\subsubsection{教学模式的作用和意义}

\textbf{1.教学模式的中介作用}

教学模式能够针对学科的特点提出理论化的教学体系,让教师可以不再盲目摸索教学,链接理论与实践,这被称为教学模式的中介作用。它表现为源于实践,也是某种理论的简化形式。

\textbf{2.教学模式的研究意义}

教学模式的研究是针对教学研究方法的一次重大进步。人们在教学研究的过程中习惯性的采用单一模式,重视对于各部分的研究,这也就忽视了各部分之间的关联。还因为忽视了实践的可实施性,导致只停留在了对于教学抽象的理解上。研究教学模式不仅可以促进教学设计的加强和教学过程的优化,还可以帮助人们去研究教学各因素之间的关联,灵活地把握教学的本质和规律。


\subsection{初中数学翻转课堂教学模式探究}

\subsubsection{翻转课堂的起源与发展}

“翻转课堂”近年来成为全球教育界关注的热点,2011年还被加拿大《环球邮报》评为“影响课堂教学的重大技术突破。

\textbf{1.翻转课堂的起源}

美国一所建立在山林地区的高中院校里面出现了两位非常有想法的教师,他们是乔纳森和萨姆斯,同时也是翻转课堂的创始人。因为各种因素导致很多学生对学习不感兴趣也很少去上课,所以他们根本掌握不了教学进度。想要将问题得以解决,教师们就通过相关的视频软件将教学内容制作成PPT进行演示,或者是录制音频,然后将录制好的讲解视频通过网络传送给学生,这样没有去上课的学生就能够在家里进行学习。

这种方式最后发展成为学生在家里就能够提前学习到课堂上的内容,这样在上课的时候就可以腾出更多的时间来帮助在学习过程中出现问题或者有困难的学生。

因此改变了传统的教学方式,以前是“教师在课堂讲课,下课布置家庭作业给学生”,而现在却是“学生提前在家学习教师的课堂内容,在上课的时候通过教师的指导来完成作业”。

教学模式改革后,这种在线教学方式也被更多没有缺席课程的学生接受,并逐渐得到大面积传播。

\textbf{2.翻转课堂的发展}

直到2010年,“翻转课堂”才因自身的优势以及影响力获得了美国甚至全世界的认同和称赞,并且还和“可汗学院”之间产生联系。在2007年之后,“翻转课堂”发展到了美国的大部分地区,并且有越来越多的地区给予关注和重视,但是还没有大面积的使用和实践,原因如下:

大部分的教师对这种方式都是比较认可的,但是想要真正地将新的教学方式开展下去,需要解决一些问题。而“可汗学院”在美国的大范围发展,则使之前所产生的问题得到了很好的处理。

2004年出现的“可汗学院”,是一名叫作塞尔曼·可汗的人开创的。其最初目的就是想要帮助学生,因此才会有了教学内容的录制,其将内容发布在了相关网站上,以帮助孩子更好地学习。之后,他还在教学内容上做了更改和添加,比如可以进行网络交流,这样也就能够让学习者更加方便地进行练习。2007年其将教学视频结合互动软件进行运用,并且通过这种方式建立了新的教学网站。2009年可汗将自己的全部心血都运用到了维护网站上面,同时也将这个特别的在线教育网站称为“可汗学院”。2010年众多资金对可汗学院进行了投资,保证由他制作的教学视频以及其他学习资料都有着很好的质量(可汗学院进而研发出了“学习控制系统”——这种系统可以对学习数据进行汇总,不但可以让学生以及教师掌握自己的进度,而且教师还能更加方便地开展翻转课堂)。

在“可汗学院”的帮助下,有更多好的教学资源免费让人使用,解决了“翻转课堂”实施过程中所遇到的问题,同时也能够让更多的教师开展“翻转课堂”。“翻转课堂”对以前传统的上课认真听教师讲课,下课完成家庭作业的方式进行了改革和“翻转”——出现了“上课之前学习教师要讲的内容,在课堂上提出自己的问题并在教师的帮助下完成作业”的模式。

刚开始出现的翻转课堂从形式上来看还是比较单调的,只有一种方式,就是通过教学视频。但是,2011年“MOOCs”的出现,让翻转课堂所体现出来的教学内容及方式发生了改变。

\subsubsection{翻转课堂的基本理论}

新课改理论是翻转课堂教学模式研究的重要指导理论,是研讨数字化教育资源环境下教学模式改革的指导思想。在中学数学翻转课堂的设计模式与应用效果研究中,下列这些理念受到重视。首先数字化教育资源主要是对教学方式以学习方式进行改善,要结合现实生活去考虑问题,展示学生自身所具备的特性,将设计思想展示出来,并且要在内容的设计上体现出知识点、过程、价值观以及方式等。同时也要把课程结构具备的选择性以及不间断性展现出来,在教育方式方面,从以前的单一性变成引导性,同时应培养学生的合作精神,使其掌握基本的生存、做人等技能,实现全面发展。

研讨“多媒体辅助教学下的翻转课堂教学模式”要建立在现代学与教的理论基础上,如元认知理论、人本理论、多元智力理论、发现学习理论、建构主义理论、学习条件理论、先行组织者理论、掌握学习理论、暗示教学理论、信息加工理论等。还有许多先进的教学方法与教学模式,这些方法与理论,从不同角度出发,对教学过程的产生的教与学进行解释,要用这些先进的教与学的理论来指导数字化教育资源开发与应用的研究工作。

“掌握学习”的主要指导思想就是要让每一个学生都能够学好,在集体教学的条件下,再结合反馈结果,帮助学生解决各类问题,同时也解决学生的课余时间问题,这样就能够保证大部分的学生能实现教师所制定的教学目标以及学习目标。

\subsubsection{翻转课堂的特点}

翻转课堂属于教学领域中被经常使用的一种方式。这种方式能够让学生和教师有更多的时间去进行沟通和互动,并且给学生营造出了一种主动的、负责任的学习氛围,这种课堂模式将构建学习和讲解进行结合,并且可以让课堂内容和信息得到有效保存,方便了学生进行复习和学习。这种课堂模式能够让学生在学习中更加积极主动,还可以充分发挥他们的个性。

\textbf{1.教学视频简单精炼}

\textbf{2.教学信息目标明确}

\textbf{3.学习流程的重建}

\textbf{4.复习检测更加简单迅速}


\subsubsection{国内外的翻转课堂}

\textbf{1.美国“翻转课堂”的特点}

(1)整理制作视频

(2)开展课堂活动

\textbf{2.美国“翻转课堂”的作用}

(1)“翻转”让学生成为学习的主人

(2)“翻转”使得教师和学生在学习中可以互动

\textbf{3.“翻转课堂”在国内的体现}

(1)转变四个方面和关注四个方面重点

(2)教师少灌输,学生多自学,教学相长,实现共赢

(3)课下学习和课堂上练习

(4)教师的讲解重复次数减少

(5)针对不同层次学生掌握情况

\subsubsection{初中数学“多媒体辅助下的翻转课堂”的基本流程}

\textbf{第一环节:课前准备}

第1步:学生进行前期微课学习,软件平台反馈给教师

第2步:教师收集学生的学习情况,形成两三个主导性问题,根据这些问题进行教学设计

主导性问题设置说明:若学生集中性问题较多,则可以采取“聚类”的方式进行设置;若学生提出的集中性问题较少,则可以结合本节课的重难点进行分类设置。

\textbf{第二环节:课上交流}

第1步:由教师主导,进行学习收获交流

第2步:针对学生的收获和问题,进行前期诊断性检测,结合信息化设备斤西瓜即时性的数据统计;根据统计情况了解学生在课堂中的学习情况,教师及时调整教学的侧重点和时间安排。

前期诊断性问题的要求说明:

\ding{172}题目数量要少,4~6道题目左右,控制完成时间

\ding{173}题目以填空题、选择题为主,便于及时统计

\ding{174}题目的设计要结合本节课的重难点,兼顾学生提出的收获和问题

第3步:结合课前设计的主导性问题推动进程,借助互动生成法、内在建构教学法进行教学。

互动生成法、内在建构法的具体要求说明:

\ding{172}教学过程中可以与小组合作教学、多元化评价等教学手段相结合

\ding{173}教师在教学过程中要进入学生的讨论当中,注重对有效教学资源的捕捉,采用语言点拨或者生成性资源的深加工方式引导学生获取对问题的深层认识

\ding{174}在教学过程中,要及时结合前期诊断性检测结果进行强化或弱化、添加或删除某些主导性问题,本环节为课堂教学的重点,时间控制在25分钟以内

第4步:结合本节课的教学内容,设置拓展延伸或者知识总结环节,对知识的学习进行拓展和系统总结,提升学生的学习能力,时间控制在8分钟以内。

第5步:针对本节课的主导性问题,设置收获性检测,要求与前期诊断性检测题目相类似,时间控制在8分钟以内。

\textbf{第三环节:课后工作}

第1步:根据信息化设备进行收获性检测的数据统计,据统计结果,了解课堂学习后学生的实际情况。

第2步:针对学生实际情况,指导掌握较好的学生回家进行后续的微课学习;对于掌握不理想的学生,利用在校时间指导学生如何结合微课进行学习,并完成备用收获性检测。




\subsection{初中数学“互动生成,内在建构”的教学模式}
\subsubsection{“互动生成,内在建构”教学模式的理念}
《数学课程标准》指出:“数学是人类文化的重要组成部分,数学素养是现代社会每一个公民应该具备的基本素养。作为促进学生全面发展教育的重要组成部分,数学教育既要使学生掌握现代生活和学习中所需要的数学知识与技能,更要发挥数学在培养人的理性思维和创新能力方面的不可替代的作用”。其强调了数学教学要实现培养学生理性思维和创新能力的育人目标。

《数学课程标准》还指出:“教学活动是师生积极参与,交往互动、共同发展的过程。教师教学应该以学生的认知发展水平和已有经验为基础,注重启发式教学和因材施教,处理好讲授与学生自主学习的关系,引导学生独立思考、主动探索、合作交流,使学生理解和掌握基本的数学知识与技能、数学思想与方法、获得基本的数学活动经验。”这说明教学活动是教学的载体,互动合作是教学手段,学生获取知识、能力、思想、方法和活动经验是教学的目的。

建构主义学习理论认为某一社会发展阶段的科学知识固然包含真理,但是并不意味着终极答案,随着社会的发展,肯定还会由更真实的解释。教学不能把知识作为预先决定了的东西教给学生,只能由他自己建构完成,以他们自己的经验来严政知识的合理性,对新知识进行分析、检验和批判,其中指出知识传授只有以学生的自主建构予以落实,才能在以知识传授为载体的教学过程中真正培养学生的能力,使旧有的知识的拓展与创新不断延续开来。

\subsubsection{“互动生成,内在建构”教学模式的基本流程}
“互动生成,内在建构”教学模式的基本流程是“主动性问题——交往互动——内在建构——生成发展”。这种教学模式特别关注教学中的主动性问题的合理下放与处理、教学资源的适时回收与加工,通过若干个精心设计的收放过程自然延展学生的知识、方法结构,拓展学生的思维空间,逐步培养学生的理性思维和整体驾驭数学的能力。以下是教学模式的逻辑流程图。

~\\
\begin{tikzpicture}
rectangle (40,6.3);
\draw [line width=1.pt] (-3,2)-- (1,2);
\draw [line width=1.pt] (1,2)-- (1,0.8);
\draw [line width=1.pt] (1,0.8)-- (-3,0.8);
\draw [line width=1.pt] (-3,0.8)-- (-3,2);
\draw (-3,1.72) node[anchor=north west] {基于已知的主动性问题};
\draw [->,line width=1.pt] (1,1.5) -- (4,1.5);
\draw (1.5,2.1) node[anchor=north west] {多维方式};
\draw (1.5,1.5) node[anchor=north west] {互动生成};

rectangle (40,6.3);
\draw [line width=1.pt] (4,2)-- (8,2);
\draw [line width=1.pt] (8,2)-- (8,0.8);
\draw [line width=1.pt] (8,0.8)-- (4,0.8);
\draw [line width=1.pt] (4,0.8)-- (4,2);
\draw (4,1.72) node[anchor=north west] {合理有效的新教学资源};
\draw [->,line width=1.pt] (8,1.5) -- (11,1.5);
\draw (8.5,2.1) node[anchor=north west] {概况归纳};
\draw (8.5,1.5) node[anchor=north west] {内在建构};

rectangle (40,6.3);
\draw [line width=1.pt] (11,2)-- (15,2);
\draw [line width=1.pt] (15,2)-- (15,0.8);
\draw [line width=1.pt] (15,0.8)-- (11,0.8);
\draw [line width=1.pt] (11,0.8)-- (11,2);
\draw (11,1.72) node[anchor=north west] {拓展创新的知识、方法};
\end{tikzpicture}

\subsubsection{“互动生成,内在建构”教学模式的策略}
\textbf{1.互动的分组分层}

教师要结合学生性格特征、学生成绩、性别等多方面因素,以小组为单位划分学生,通常每班分成4大组,然后再分为10~12小组,一组3或4名学生,安排座次。要充分考虑到学生的个体差异性,在分组时,要平衡各小组的实力,确保小组内成员的和谐。为保证分组认同性、合理性,一般在某一学段测试基础上进行分组;分组后根据学生学科成绩状况,每小组设一学科小组长,每一大组设一学科负责人,并根据学生学科成绩确定A、B、C、D四个学习层次的人员。

\textbf{2.主动性问题设计}

问题如果简单、细碎,学生通常不需要深入思考就能作出正确答案,这样就会频繁产生教师和学生的重复互动,产生无效劳动多,课堂效率低的问题。因此细碎的教学设计一定会造成封闭、死板的教学,揭示了教师利用制定学习方案,按部就班的让学生机械性的学习知识的现象。

与之相对的是“主动性问题”,它是调动学生主观意识,激发学生渴望学习的态度,从而培养学生进行深度学习,养成良好学习习惯。“主动性问题”设计要关注学生基础性状态。教师对学生潜在状态和发展需要解读越是清晰,问题设计的“整合”程度对于学生来说就是越具有切入性,越能够发现和捕捉不同学生解决问题过程中不同的思维状态,从而生成有效的教学资源。

“主动性问题”设计,要充分考虑学生在学习中表现的状态,获取知识的途径,比如通过选择、发现、重组等方式。教师要进行综合性评价,让学生的潜力充分发挥出来,教师评价越准确,取得的效果越好,越有利于建立师生、生生良好的互动关系,帮助学生塑造认知结构,并且提升思维水平。

“主动性问题”,具有一定的复杂性,因此在教学方面有难度,如何掌握学生的学习能力,让学生有效利用学过的知识处理问题,需要一定的思考和时间。这是教师教学的新课题,首先要给予学生一定的时间去独立思考和解决问题。其次,要给予学生充分的尊重,相信他们的能力。

\textbf{3.互动生成的落实}

实施此教学模式的关键在于实施教学过程中是否做好“互动”“生成”,即是否做好、做透“放”与“收”这两个环节。

所谓“放”,首先就是实行放权,将课堂主动权交给学生,让学生自己处理问题,可以独立思考,也可以团队合作。不再是教师单方面地传授解题思路,这样可以确保每一名学生都参与到解决问题中来。其次,放弃代替思维,让教学工作由个别优等生,面向全部学生,将班级的学习气氛调动起来,让学生之间的交流沟通多起来,这样有利于加强学习知识的传播,使得信息流通于每个学生之间,不存在阻塞现象。最后,放弃教学普遍性,让教学工作更具有针对性,有利于细致地解读学生的内心,让教学工作更加贴合学生的意愿,从而提高学生的积极性。

对于“收”而言,是指及时的进行教学信息收集,主要针对学生处理问题时产生的相关信息进行回收,形成教师与学生的互动资源,也可以成为学生之间的互动资源,使教学工作能够结合这些资源,在其基础上更好地利用。“收”可以解决教学的盲点,让教师从只抓教学成果,转变为重视学生对问题的处理中来,要帮助他们解决遇到的问题,从错误解读中纠正过来,从而将学生的价值发挥出来,而不是作为一种教学资源,将教学作为走过场。只有将学生的解题过程、互动交流作为教学的重要指标,才能真正促使教师将关注点转移到学生思维运用中来,将培养和提升学生思维能力作为教学重点,从而唤醒学生的学习积极性,促使学生的思维能力不断得到提升,从而更好地解决问题。

对于教学形式而言,通常分为两个层次。第一,收集学生个体差异性的相关信息,即每个学生的思维能力,对于处理问题时的反应与表现,将这些信息收集上来,并且采用小组形式,开展讨论,让学生进行互动。第二,在此基础上,进行全班交流,从而形成师生互动。    




\end{document}