\documentclass{article} %oneside消除奇偶页出现的空页问题,页眉也会随之改变
\usepackage[zihao=-4]{ctex} %引入中文宏包
\usepackage{ulem}%提供\uline命令引入高低一致的下划线
\usepackage{amsmath}%数学公式排版
\usepackage{amssymb}%用命令 \mathbb
\usepackage{geometry}%编辑页面边距
\usepackage{pifont}%编辑带圈数字
\usepackage{fancyhdr}%风格
\usepackage[backref]{hyperref}%超链接
\usepackage{color}%字体颜色
\usepackage{extarrows}%输入带文字的长等号
\usepackage[dvipsnames,svgnames]{xcolor}
\usepackage[strict]{changepage} % 提供一个 adjustwidth 环境
\usepackage{framed} % 实现方框效果
\usepackage{amsthm}
\usepackage{rotating}

% ----------------插入代码----------------
\usepackage{listings}
\definecolor{dkgreen}{rgb}{0,0.6,0}
\definecolor{gray}{rgb}{0.5,0.5,0.5}
\definecolor{mauve}{rgb}{0.58,0,0.82}
\lstset
{
    frame=tb,
    language=Matlab,
    aboveskip=3mm,
    belowskip=3mm,
    showstringspaces=false,
    columns=flexible,
    basicstyle={\small\ttfamily},
    numbers=left,%设置行号位置none不显示行号
    %numberstyle=\tiny\courier, %设置行号大小
    numberstyle=\tiny\color{gray},
    keywordstyle=\color{blue},
    commentstyle=\color{dkgreen},
    stringstyle=\color{mauve},
    breaklines=true,
    breakatwhitespace=true,
    escapeinside=``,%逃逸字符(1左面的键),用于显示中文例如在代码中`中文...`
    tabsize=4,
    extendedchars=false %解决代码跨页时,章节标题,页眉等汉字不显示的问题
}


\hypersetup{colorlinks=true,linkcolor=cyan,anchorcolor=megenta,citecolor=yellow}
\geometry{right=2.5cm,left=2.5cm,top=2.5cm,bottom=2.5cm}
\renewcommand{\headrulewidth}{0pt}%去掉页眉分划
\pagestyle{fancy}

% ------------------文字底色------------------
\newtheorem{example}{}
\newtheorem{examp}{}
\definecolor{egshade}{rgb}{0.96,0.96,0.96}% 
\newenvironment{eg}{%
\def\FrameCommand{%
\hspace{1pt}%
{\color{Gray}\vrule width 2pt}%竖线颜色
{\color{egshade}\vrule width 4pt}%
\colorbox{egshade}%
}%
\MakeFramed{\advance\hsize-\width\FrameRestore}%
\noindent\hspace{-4.55pt}% 
\begin{adjustwidth}{}{7pt}%
\vspace{2pt}\vspace{2pt}%
\normalfont %环境字体设置
}
{%
\vspace{2pt}\end{adjustwidth}\endMakeFramed%
}

\newtheorem{warning}{}  
\definecolor{washade}{rgb}{0.99,0.95,0.94} % 方框底部颜色
\newenvironment{wa}{%
\def\FrameCommand{%
\hspace{1pt}%
{\color{LightCoral}\vrule width 2pt}%竖线颜色
{\color{washade}\vrule width 4pt}%
\colorbox{washade}%
}%
\MakeFramed{\advance\hsize-\width\FrameRestore}%
\noindent\hspace{-4.55pt}% disable indenting first paragraph
\begin{adjustwidth}{}{7pt}%
\vspace{2pt}\vspace{2pt}%
\normalfont %
}
{%
\vspace{2pt}\end{adjustwidth}\endMakeFramed%
}



% ----------------------正文------------------------------------------------------------
\begin{document} 
\centerline{\LARGE \textbf{数学软件} }\par 

% ---------------23.02.21----------------
\noindent \Large \section*{23.02.21} \par \normalsize
\begin{eg}
    1.计算$\sum \limits _{i=1}^{100} 2^{i}$ \par
\end{eg}
\begin{lstlisting}
    i = 1 : 100
    a = 2.^i    % i 是向量,所以要用点乘
    sum(a)
\end{lstlisting}


\begin{eg}
    2.将数取整的几种方式
\end{eg}
\begin{lstlisting}
    a = 5.3
    floor(a)   %向下取整
    ceil(a)    %向上取整
    round(a)    %四舍五入
\end{lstlisting}

\begin{eg}
    3.自定义函数
\end{eg}
\begin{lstlisting}
    function y = gab(x)
    y = log2(abs(x^2 - 5)) + (sin(x) + cos(x))^6; 
    %新建一个 gab.m 文件将函数自定义,在命令行窗口可直接求得gab(x)的值
\end{lstlisting}
\begin{lstlisting}
    function y = gaa(x)
    y = 100*(x(2) - x(1))^2 + x(1)*x(2);
    %两个自变量的情况,命定行窗口gaa([x1,x2])
\end{lstlisting}

\begin{eg}
    4.取值问题
\end{eg}
\begin{lstlisting}
    i = 4 : 2 : 10  %从4到10,步长为2的所有数
    linspace(1 ,10 ,5)  %从1到10等长取5个数
    logspace(1 ,1024 ,10)   %创建行向量,从10^1到10^1024,生成总数为10的等比数列
\end{lstlisting}

\begin{eg}
    5.矩阵的常见使用函数
\end{eg}
\begin{lstlisting}
    a = [1 2 3 ;4 5 6 ;7 8 9]
    a'  %转置
    a( : , 3) = [ ]   %去除第3列
    a( 2, : ) = [ ]   %去除第2行
    det(a)  %求行列式
    inv(a)  %求逆
    rank(a)  %求秩
    format rat  %用分数形式表示数值
\end{lstlisting}
\par
\newpage
% -----------------23.02.24------------------
\noindent \Large \section*{23.02.24} \par \normalsize
\begin{eg}
    1.输入矩阵
\end{eg}
\begin{lstlisting}
    %----常规----
    ones(m,n)   %全1
    zeros(m,n)   %全0
    eye(m,n)    %单位矩阵
    %----拼接----
    >> C = [A B]    %行

C =

     1     1     0     0
     1     1     0     0

    >> C = [A ;B]   %列
 
C =

     1     1
     1     1
     0     0
     0     0

    %----直接输入----
    a = [1 2 3 ;4 5 6 ;7 8 9]
\end{lstlisting}

\begin{eg}
    2.数组的运算

    当两个数组具有相同的维数时,运算可按元素对元素的方式进行,不同大小或维数的数组不能进行运算P248
\end{eg}

\begin{eg}
    3.空矩阵的作用
\end{eg}
\begin{lstlisting}
    %----去掉某行、列----
    a = [1 2 3 ;4 5 6 ;7 8 9]
    a( : , 3) = [ ]   %去除第3列
    a( 2, : ) = [ ]   %去除第2行  
\end{lstlisting}

\begin{eg}
    4.矩阵中元素的操作
\end{eg}
\begin{lstlisting}
    %----提取某行、列----
    a(1 : 2, : )    %提取1、2行,所有列
    a(2 : 2 : 100 , : )    %提取偶数行
    a( : , 1 : 2 : 99)    %提取奇数行
    a( : , 99 : -2 : 1)    %提取奇数行逆序
\end{lstlisting}

\begin{eg}
    5.矩阵的运算
\end{eg}
\begin{lstlisting}
    A + B   %矩阵加法
    A * B   %矩阵惩罚
    det(A)  %方阵的行列式
    inv(A)  %方阵的逆
    rank(A)  %求秩
    eig(A)  %方阵特征值与特征向量
\end{lstlisting}

\begin{wa}
    行列式的原始定义

    \noindent 特征值、特征向量的代数和几何意义
\end{wa}

\begin{eg}
    6.运算符,详见P248
\end{eg}

\begin{eg}
    6.控制流结构
\end{eg}

\begin{wa}
    例:计算$\sum \limits _{i=1}^{100} i$ 
\end{wa}
\begin{lstlisting}
    %----for----
    s = 0
    for i = 1 : 100
        s = s + i
    end
    %----while----
    s = 0 , i = 1
    while i <= 100
        s = s + i
        i = i + 1
    end
\end{lstlisting}

\begin{wa}
    例:$1.005^{n} \geqslant 2$,求$n$的最小值
\end{wa}
\begin{lstlisting}
    %----teacher----
    n = 0   %这里取n = 1也不影响结果
    while 1.005^n < 2 
        n = n + 1
    end

    %----me----
    n = 0
    while n < log(2)/log(1.005)
        n = n + 1
    end
\end{lstlisting}

\begin{wa}
    例:计算$\sum \limits _{i=1}^{100} \dfrac{6^{i - 1}}{7^{C(i)}}$
    $$C(i) =
    \begin{cases}
        2i & i \leqslant 5 \\
        i^{2} & i > 5
    \end{cases}
    $$
\end{wa}
\begin{lstlisting}
    %----if-else----
    s = 0 
    for i = 1 : 100
        if i <= 5
            t = 6^( i - 1 ) / 7^( 2*i ) 
        else
            t = 6^( i - 1 ) / 7^( i^2 ) 
        end
            s = s + t 
    end
\end{lstlisting}

\begin{wa}
    例:设
    $$f(x) = 
    \begin{cases}
        x^{2} + 1 & x > 1\\
        2x & x \leqslant 1
    \end{cases}
    $$
    \indent 求$f(2)$,$f(1)$
\end{wa}
\begin{lstlisting}
    %----新建M文件----
    function y = fun1(x)
    if x > 1
        y = x^2 + 1
    else
        y = 2*x
    end
\end{lstlisting}

\begin{wa}
    例:设
    $$f(x) = 
    \begin{cases}
        x^{2} + 1 & x > 1\\
        2x & 0 < x \leqslant 1 \\
        x^{3} & x \leqslant 0
    \end{cases}
    $$
    \indent 求$f(2)$,$f(0.5)$,$f(-1)$
\end{wa}
\begin{lstlisting}
    %----新建M文件----
    function y = fun1(x)
    if x > 1
        y = x^2 + 1
    elseif 0 < x <= 1
        y = 2 * x
    else
        y = x^3
    end
\end{lstlisting}

\begin{wa}
    作业
\end{wa}
\begin{lstlisting}
    function y = sort(x)
    for i = 1 : 9
        for j = 1 : 10-i
            if x( j ) > x( j + 1 )
                a = x( j );
                x( j ) = x( j + 1 ); 
                x( j + 1 ) = a ;
            end
        end
        y = x ;
    end
\end{lstlisting}
\begin{lstlisting}
    function m =maxele( a ) 
    ii = 1 ;
    jj = 1 ;
    i = 1 ; j = 1 ; 
    m = a( i  ,  j ) ;
    for i = 1 : 4   
        for j = 1 : 5
            if m < a( i , j )
                m =  a( i , j ) ;
                ii = i ;
                jj = j ;
            end
        end
    end
    ii
    jj
    end
\end{lstlisting}
\begin{lstlisting}
    function y = fac(x)
    a = 1;
    y = 0;
    for n = 1 : x
            a = a * n ;
            y = y + a ;      
    end 
    end
\end{lstlisting}
\begin{lstlisting}
    function y = balls(x)
    y = 0 ;
    s = 100 ;
    for a = 1 : x
        s = s / 2 ;
        y = y + 2*s + s ;
    end
    s
\end{lstlisting}
\begin{lstlisting}
    function y = cula (a,b)
    y = a^2 + sin(a*b) + 2*b ;
\end{lstlisting}


\newpage
% -----------------23.02.28------------------
\noindent \Large \section*{23.02.28} \par \normalsize
\begin{wa}
    “作业1”习题讲解
\end{wa}
\begin{lstlisting}
    function y = sorted1( x )
    a = randperm( x );
    n = length( a ) ;
    for i = 1 : n - 1
        for j = n : -1 : i + 1    %起泡法从后往前排序
            if a( j ) < a( j - 1 )
                b = a( j - 1 ) ;
                a( j - 1 ) = a( j ) ;
                a( j ) = b ;
            end
            y = a
        end
    end
\end{lstlisting}
\begin{lstlisting}
    function y = maxele1( m , n )
    a = rand( m , n ) ;
    ma = -inf ; w1 = [ ] ;
    for i = 1 : m
        for j = 1 : n
            if a( i , j ) >= ma
                ma = a( i , j ) ;
                w = [ i , j ] ;
                w1 = [ w1 ; w ] ;   %当有相同元素出现多个坐标的常见处理方法
            end
        end
    end
    y = ma ;
\end{lstlisting}

\begin{eg}
    1.赋值语句
\end{eg}
\begin{lstlisting}
    for i = 1 : 10
        y = sin( i * pi / 180 ) ;
    end
\end{lstlisting}

\begin{eg}
    2.内存变量的删除和修改
\end{eg}
\begin{lstlisting}
    save big.mat    %储存
    load big.mat    %调用
\end{lstlisting}

\begin{eg}
    3.常用数学函数
\end{eg}
\begin{lstlisting}
    sin(n * pi / 180)  %弧度转换
    abs(n)    %实数的绝对值、复数的模、字符串的ASCII码值
    fix(n)    %函数取整
    floor(n)
    ceil(n)
    round(n)
    rem(x , y)    %当x 、y不同号,结果不同
    mod(x , y)    %x、y必须为相同大小的实矩阵或为标量
\end{lstlisting}

\begin{eg}
    4.矩阵元素
\end{eg}
\begin{lstlisting}
    A(3 , 2) = 200    %通过下标引用矩阵元素
    A = [ 1 , 2 , 3 ; 4 , 5 ,6 ] ;      %采用矩阵元素序号引用元素
    A(3)
    ans = 2     %在MATLAB中,列优先
\end{lstlisting}

\begin{eg}
    5.矩阵拆分
\end{eg}
\begin{lstlisting}
    A( : , j )      %取第j列全部元素
    A( i , : )      %取第i行全部元素
    A( i : i + m , : )      %第i到第i+m行全部元素
    A( : , k : k + m)       %第k到第k+m列全部元素
    A( 2 : end , : )    %第2到最后一行全部元素
\end{lstlisting}

\begin{eg}
    6.特殊矩阵
\end{eg}
\begin{lstlisting}
    zeros(n)    %零矩阵
    ones(n)   %幺矩阵
    eye(n)  %单位阵
    rand(n) %0~1间随机均匀分布矩阵
    randn(n)    %产生均值为0,方差为1的标准正态分布矩阵
    reshape(A , m , n)  %将矩阵A重排为m*n
\end{lstlisting}

\begin{wa}
    建立随机矩阵:
    
    (1)在区间$[20,50]$内均匀分布的$5$阶随机矩阵
    
    (2)均值为0.6,方差为0.1的5阶正态分布随机矩阵
\end{wa}
\begin{lstlisting}
    x = 20 + (50 - 20)*rand(5)
    y = 0.6 + sqrt(0.1)*randn(5)
\end{lstlisting}

\begin{eg}
    7.魔法阵、$Vandermonde$阵、$Pascal$阵
\end{eg}

\begin{wa}
    将101-125填入一个5行5列的表格中,使其每行每列及对角线的和均为565
\end{wa}
\begin{lstlisting}
    M = 100 + magic(5)
\end{lstlisting}

\begin{eg}
    8.对角阵、三角阵
\end{eg}




\newpage
% -----------------23.03.03------------------
\noindent \Large \section*{23.03.03} \par \normalsize
\begin{wa}
    作业2:用diag构造一个100*100的矩阵
    $$A =     
    \begin{bmatrix}
        1 & 2 & \dots & 100 \\
        101 & 102 & \dots & 200 \\
        \vdots & \vdots & \ddots & \vdots \\
        9901 & 9902 & \dots & 10000 
    \end{bmatrix}$$
\end{wa}
\begin{lstlisting}
    ones(100)*diag(1:100) + diag(0 : 100 : 9900)*ones(100)
    ones(100)*diag(1:100) + 100 * diag(0 : 99)*ones(100)
\end{lstlisting}

\begin{eg}
    1.矩阵的转置与旋转
\end{eg}
\begin{lstlisting}
    A = [1 , 2 , 3 ;
        4 , 5 , 6 ]
    A'  %转置
    rot90( A , 1)   %向右旋转90°
    rot90( A , 2)   %向右旋转180°
%---------------------------------------
    >> A'
    rot90( A , 1)
    rot90( A , 2)

    ans =

         1     4
         2     5
         3     6


    ans =

         3     6
         2     5
         1     4


    ans =

         6     5     4
         3     2     1
\end{lstlisting}

\begin{eg}
    2.矩阵的翻转
\end{eg}
\begin{lstlisting}
     A = [1 , 2 , 3 , 4;
        5 , 6 , 7 , 8 ]
    fliplr(A)   %左右翻转,也可通过设置步长为负数,即A = A(:,4:-1:1)
    flipud(A)   %上下翻转

    A =

         1     2     3     4
         5     6     7     8


    ans =

         4     3     2     1
         8     7     6     5


    ans =

         5     6     7     8
         1     2     3     4
\end{lstlisting}

\begin{eg}
    3.矩阵的伪逆
    当矩阵不为方阵或行列式为零
\end{eg}
\begin{lstlisting}
    pinv(A)
\end{lstlisting}

\begin{eg}
    4.向量的3种常用范数及其计算函数
\end{eg}
\begin{lstlisting}
    norm(A,2)   %计算向量A的2—范数
    norm(A,1)   %计算向量A的1—范数
    norm(A,inf)    %计算向量A的$\infty$—范数
\end{lstlisting}

\begin{eg}
    5.矩阵的特征值与特征向量
\end{eg}
\begin{lstlisting}
    E = eig(A)    %求矩阵A的全部特征值构成向量E
    [V , D] = eig(A)   %求矩阵A的全部特征值,构成对角阵D,求A的特征向量构成V的列向量
%-------------------------------------------------
    >>A = [1 , 2 , 3 ; 4 , 5 ,6 ; 7 , 8 , 9 ];
    [V , D] = eig(A)
    V =

       -0.2320   -0.7858    0.4082
       -0.5253   -0.0868   -0.8165
       -0.8187    0.6123    0.4082


    D =

       16.1168         0         0
             0   -1.1168         0
             0         0   -0.0000

\end{lstlisting}

\begin{wa}
    用特征值的方法解方程
    $$3x^{5} - 7x^{4} + 5x^{2} + 2x - 18 = 0$$
\end{wa}
\begin{lstlisting}
    p = [ 3 , -7 , 0 , 5 , 2 , -18 ];
    A = compan(p);  %A的伴随矩阵
    x1 = eig(A);    %求A的特征值
    x2 = roots(p);  %求多项式p的零点
\end{lstlisting}

\begin{eg}
   矩阵的秩:

    1.最大阶非零子式阶数

    2.线性无关的向量个数

    3.阶梯形矩阵非零行个数 
\end{eg}

\begin{eg}
    7.字符串

    MATLAB将字符串当做一个行向量,每个元素对应一个字符
\end{eg}
\begin{lstlisting}
    m = 'Yangtze';
    a = abs(m)
    a = double(m)
    char(a)
%---------------------------------------
    >>a =

    89    97   110   103   116   122   101


    a =

    89    97   110   103   116   122   101


    ans =

    Yangtze

\end{lstlisting}

\begin{wa}
    建立一个字符串向量,对该向量进行如下处理

    1.取第1~5个字符组成的子字符串

    2.将字符串倒过来重新排列

    3.将字符串中的小写字母边成相应的大写字母

    4.统计字符串中小写字母的个数
\end{wa}
\begin{lstlisting}
    ch = 'ABc123d4ceFg56H9';
    subch = ch(1 : 5)   %取子字符串
    revch = ch(end : -1 : 1)    %将字符串倒排
    k = find(ch >= 'a' & ch <= 'z');    %找小写字母位置
    ch(k) = ch(k) - ('a' - 'A');    %变为大写字母
    char(ch)
    length(k)   %统计小写字母个数
\end{lstlisting}

\begin{eg}
    8.结构数据和单元数据
\end{eg}
\begin{lstlisting}
    a(1).x1 = '123';    %给a中任意一个元素增加一个成员x4
    a = rmfield(a, x4)  %删除x4    
\end{lstlisting}

\begin{eg}
    9.单元数据
\end{eg}
\begin{lstlisting}
    b{ 3 , 3};    %引用单元矩阵元素
    celldisp(b)    %显示整个单元矩阵
\end{lstlisting}

\begin{eg}
    10.稀疏存储

    仅存储非零元的值和位置
\end{eg}



\newpage
% -----------------23.03.07------------------
\noindent \Large \section*{23.03.07} \par \normalsize
\begin{wa}
    作业2:试用简单矩阵生成下面的矩阵
$$
\begin{bmatrix}
    % 10 & 10 & 10 & 10 & 10 & 10 & 10 & 10 & 10 & 10 & 10 & 10 & 10 & 10 & 10 & 10 & 10 & 10 & 10 & 10 & 10 \\
    % 10 & 9 & 9 & 9 & 9 & 9 & 9 & 9 & 9 & 9 & 9 & 9 & 9 & 9 & 9 & 9 & 9 & 9 & 9 & 9 & 10 \\
    % 10 & 9 & 8 & 8 & 8 & 8 & 8 & 8 & 8 & 8 & 8 & 8 & 8 & 8 & 8 & 8 & 8 & 8 & 8 & 9 & 10 \\
    % 10 & 9 & 8 & 7 & 7 & 7 & 7 & 7 & 7 & 7 & 7 & 7 & 7 & 7 & 7 & 7 & 7 & 7 & 8 & 9 & 10 \\
    % 10 & 9 & 8 & 7 & 6 & 6 & 6 & 6 & 6 & 6 & 6 & 6 & 6 & 6 & 6 & 6 & 6 & 7 & 8 & 9 & 10 \\
    % 10 & 9 & 8 & 7 & 6 & 5 & 5 & 5 & 5 & 5 & 5 & 5 & 5 & 5 & 5 & 5 & 6 & 7 & 8 & 9 & 10 \\
    % 10 & 9 & 8 & 7 & 6 & 5 & 4 & 4 & 4 & 4 & 4 & 4 & 4 & 4 & 4 & 5 & 6 & 7 & 8 & 9 & 10 \\
    % 10 & 9 & 8 & 7 & 6 & 5 & 4 & 3 & 3 & 3 & 3 & 3 & 3 & 3 & 4 & 5 & 6 & 7 & 8 & 9 & 10 \\
    % 10 & 9 & 8 & 7 & 6 & 5 & 4 & 3 & 2 & 2 & 2 & 2 & 2 & 3 & 4 & 5 & 6 & 7 & 8 & 9 & 10 \\
    % 10 & 9 & 8 & 7 & 6 & 5 & 4 & 3 & 2 & 1 & 1 & 1 & 2 & 3 & 4 & 5 & 6 & 7 & 8 & 9 & 10 \\
    % 10 & 9 & 8 & 7 & 6 & 5 & 4 & 3 & 2 & 1 & 0 & 1 & 2 & 3 & 4 & 5 & 6 & 7 & 8 & 9 & 10 \\
    % 10 & 9 & 8 & 7 & 6 & 5 & 4 & 3 & 2 & 1 & 1 & 1 & 2 & 3 & 4 & 5 & 6 & 7 & 8 & 9 & 10 \\
    % 10 & 9 & 8 & 7 & 6 & 5 & 4 & 3 & 2 & 2 & 2 & 2 & 2 & 3 & 4 & 5 & 6 & 7 & 8 & 9 & 10 \\
    % 10 & 9 & 8 & 7 & 6 & 5 & 4 & 3 & 3 & 3 & 3 & 3 & 3 & 3 & 4 & 5 & 6 & 7 & 8 & 9 & 10 \\
    % 10 & 9 & 8 & 7 & 6 & 5 & 4 & 4 & 4 & 4 & 4 & 4 & 4 & 4 & 4 & 5 & 6 & 7 & 8 & 9 & 10 \\
    % 10 & 9 & 8 & 7 & 6 & 5 & 5 & 5 & 5 & 5 & 5 & 5 & 5 & 5 & 5 & 5 & 6 & 7 & 8 & 9 & 10 \\
    % 10 & 9 & 8 & 7 & 6 & 6 & 6 & 6 & 6 & 6 & 6 & 6 & 6 & 6 & 6 & 6 & 6 & 7 & 8 & 9 & 10 \\
    % 10 & 9 & 8 & 7 & 7 & 7 & 7 & 7 & 7 & 7 & 7 & 7 & 7 & 7 & 7 & 7 & 7 & 7 & 8 & 9 & 10 \\
    % 10 & 9 & 8 & 8 & 8 & 8 & 8 & 8 & 8 & 8 & 8 & 8 & 8 & 8 & 8 & 8 & 8 & 8 & 8 & 9 & 10 \\
    % 10 & 9 & 9 & 9 & 9 & 9 & 9 & 9 & 9 & 9 & 9 & 9 & 9 & 9 & 9 & 9 & 9 & 9 & 9 & 9 & 10 \\
    % 10 & 10 & 10 & 10 & 10 & 10 & 10 & 10 & 10 & 10 & 10 & 10 & 10 & 10 & 10 & 10 & 10 & 10 & 10 & 10 & 10 
    
    10 & \dots & 10 & 10 & 10 & \dots & 10 \\
     & \ddots & \vdots & \vdots & \vdots & \begin{rotate}{60} $\ddots$ \end{rotate} \\
    10 & \cdots & 1 & 1 & 1 & \cdots & 10 \\
    10 & \cdots & 1 & 0 & 1 & \cdots & 10 \\
    10 & \cdots & 1 & 1 & 1 & \cdots & 10 \\
    & \begin{rotate}{70} $\ddots$ \end{rotate} & \vdots & \vdots & \vdots & \ddots\\
    10 & \cdots & 10 & 10 & 10 &\cdots & 10
\end{bmatrix}
$$
\end{wa}
\begin{lstlisting}
    n = 11;
    a = ones(n);    %生成一个全1矩阵
    a1 = triu(a*diag(0 : n-1));    %生成一个0到10的上三角阵
    a2 = a1 + triu(a1,1)';     %生成目标矩阵的右下角部分,从0到10 
    a2 = [fliplr(a2),a2];    %矩阵向左翻转,拼接
    a2(: , n) = [];    %去掉中间重复的一列
    a2 = [flipud(a2),a2];   %向上翻转,拼接
    a2(n , :) = [];    %去掉中间重复的一行
\end{lstlisting}

\begin{eg}
    1.input命令
\end{eg}
\begin{wa}
    将华氏温度$f$转化为摄氏温度$c$
\end{wa}
\begin{lstlisting}
    f = input('Input Fahrenheit Temperature:');
    c = 5*(f - 32) / 9
\end{lstlisting}
\begin{wa}
    输入$x$,$y$值,并将它们的值互换后输出
\end{wa}
\begin{lstlisting}
    x = input('x =');
    y = input('y =');
    z = x;
    x = y;
    y = z;
    disp(x);
    disp(y);
\end{lstlisting}

\begin{eg}
    2.disp数据输出
\end{eg}
\begin{wa}
    求一元二次方程$ax^{2} + bx + c = 0$的根
\end{wa}
\begin{lstlisting}
    a = input('a = ');
    b = input('b = ');
    c = input('c = ');
    d = b*b - 4*a*c;
    x = [(-b + sqrt(d)) / (2*a) , (-b - sqrt(d)) / (2*a)];
    disp(['x1 = ' , num2str(x(1)) , 'x2 = ' , num2str(x(2))])   %num2str(x(2))字符串,无法再进行加减
\end{lstlisting}

\begin{eg}
    3.switch语句
\end{eg}
\begin{lstlisting}
    price = input('input price : ');
    switch fix(price/100)
        case {0 , 1}
            rate = 0;
        case {2 , 3 , 4}
            rate = 3 / 100;
        case num2cell(5 : 9)
            rate = 5 / 100;
        otherwise
            rate = 8 / 100;
    end
    price = price*(1 - rate)
\end{lstlisting}

\begin{eg}
    4.try语句
\end{eg}
\begin{lstlisting}
    try
        a = a + q
        disp('a')
    catch
        disp('error')
        a = nan
    end
\end{lstlisting}

\begin{eg}
    5.pause暂停时间间隔
\end{eg}
\begin{lstlisting}
    a = 0 ;
    for k = 1 : 10
        a = a + 1 
        pause(1)    %输出一个值间隔一秒
    end
\end{lstlisting}

\begin{eg}
    6.break和continue,停止和跳过
\end{eg}
\begin{lstlisting}
    a = 0 ;
    for k = 1 : 10
        a = a + 1 
        pause(1)
    end
\end{lstlisting}

\begin{wa}
    输出全部水仙花数
\end{wa}
\begin{lstlisting}
    for m = 1 : 999
        m1 = fix(m / 100);
        m2 = rem(fix(m / 10) , 10);
        m3 = rem(m , 10);
        if m == m1*m1*m1 + m2*m2*m2 + m3*m3*m3
            disp(m)
        end
    end
\end{lstlisting}

\begin{wa}
    从键盘输入若干个数,当输入0时结束输入,求这些数的平均值和它们的和
\end{wa}
\begin{lstlisting}
    sum = 0;
    cnt = 0;
    val = input('please input a number : ');
    while (val ~= 0)
        sum = sum + val;
        cnt = cnt + 1;
        val = input('please input a number : ');
    end
    if (cnt > 0)
        sum
        mean = sum / cnt
    end
\end{lstlisting}

\begin{eg}
    8.循环嵌套
\end{eg}
\begin{wa}
    求1到500之间的全部完数(这个数等于它的各个真因子之和)
\end{wa}
\begin{lstlisting}
    for m = 1 : 500
        s = 0;
        for k = 1 : m/2
            if rem(m , k) == 0
                s = s + k;
            end
        end
        if m == s
            disp(m);
        end
    end
\end{lstlisting}


\newpage
% -----------------23.03.10------------------
\noindent \Large \section*{23.03.10} \par \normalsize
\begin{wa}
    作业3
\end{wa}
\begin{lstlisting}
    m = fix(100*rand) ;
    k = 1 ;
    for k = 1 : 7 
        k = k + 1 ;
        n = input('please input a number:');
        if m == n
            disp('won!');
            break ;
        elseif n > m
            disp('high!');
        else
            disp('low!');
        end
    end
    disp('game over!');
    % ------------------------------------------
    m = input('please input the last number:');
    n = 2 : m;
    for i = 2 : sqrt(m)   %根号m是m最大的因子,提高运算效率
        for j = 2*i : i : m %以倍数为步长
            n(j) = 0;
        end
    end
    a = find(n ~= 0);
    disp(n(a));
    % ------------------------------------------
    n = input('please input the last number:');
    a = zeros(n, 1);    %将a设置为n行1列的空矩阵,元素全为0
    a(1) = 1;
    a(2) = 2;
    for k = 3 : n
        a(k) = a(k-1) + a(k-2);
    end
    disp(a(n));
\end{lstlisting}
\begin{eg}
    1.plot \ 点、线
\end{eg}
\begin{lstlisting}
    x = linspace(0,2*pi,30);
    y = sin(x);
    z = cos(x);
    plot(x,y,'r' ,x,z,'go')
\end{lstlisting}

\begin{eg}
    2.ezplot \ 隐函数绘图、参数方程绘图
\end{eg}
\begin{lstlisting}
    ezplot('sin(x)' , [0 , pi])
    ezplot('cos(t)^3' , 'sin(t)^3' , [0 , 2*pi])    %星形图
    ezplot('exp(x) + sin(x*y)' , [-2 , 0.5 ,0 , 2])  %[-2,0.5][0,2]上的隐函数图
\end{lstlisting}

\begin{eg}
    3.fplot \ 自定义函数
\end{eg}
\begin{lstlisting}
    fplot('tanh' , [-2 , 2])
    fplot('[tanh(x) , sin(x) , cos(x)]' , 2*pi*[-1 , 1])
\end{lstlisting}

\begin{eg}
    4.坐标变换
\end{eg}
\begin{lstlisting}
    x = logspace(-1, 2);    %logspace生成对数间隔向量
    loglog(x, exp(x) , '-s')    %全对数函数
    grid on
% ---------------------------------
    x = 0 : .1 : 10 ;
    semilogy(x , 10.^x)     %半对数函数
\end{lstlisting}

\begin{eg}
    5.空间曲线 —— 单条曲线
\end{eg}
\begin{lstlisting}
    t = 0 : pi/50 : 10*pi;
    plot3(sin(t) , cos(t) , t)
\end{lstlisting}

\begin{eg}
    6.空间曲线 —— 多条曲线
\end{eg}
\begin{lstlisting}
    x = -3 : .1 : 3 ;
    y = 1 : .1 : 5 ;
    [X , Y] = meshgrid(x, y);
    Z = (X + Y) .^ 2;
    plot3(X , Y , Z)    %线性曲面
    surf(X, Y, Z)   %曲面
    shading flat    %平滑
    mesh(X, Y, Z)   %网格曲面
    

    Z = peaks;
    surf(Z)
\end{lstlisting}





















\end{document}