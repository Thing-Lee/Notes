\documentclass{article} %oneside消除奇偶页出现的空页问题,页眉也会随之改变
\usepackage[zihao=-4]{ctex} %引入中文宏包
\usepackage{ulem}%提供\uline命令引入高低一致的下划线
\usepackage{amsmath}%数学公式排版
\usepackage{amssymb}%用命令 \mathbb
\usepackage{geometry}%编辑页面边距
\usepackage{pifont}%编辑带圈数字
\usepackage{fancyhdr}%风格
\usepackage[backref]{hyperref}%超链接
\usepackage{color}%字体颜色
\usepackage{extarrows}%输入带文字的长等号
\usepackage[dvipsnames,svgnames]{xcolor}
\usepackage[strict]{changepage} % 提供一个 adjustwidth 环境
\usepackage{framed} % 实现方框效果
\usepackage{amsthm}

% ----------------插入代码----------------
\usepackage{listings}
\definecolor{dkgreen}{rgb}{0,0.6,0}
\definecolor{gray}{rgb}{0.5,0.5,0.5}
\definecolor{mauve}{rgb}{0.58,0,0.82}
\lstset
{
    frame=tb,
    language=Matlab,
    aboveskip=3mm,
    belowskip=3mm,
    showstringspaces=false,
    columns=flexible,
    basicstyle={\small\ttfamily},
    numbers=left,%设置行号位置none不显示行号
    %numberstyle=\tiny\courier, %设置行号大小
    numberstyle=\tiny\color{gray},
    keywordstyle=\color{blue},
    commentstyle=\color{dkgreen},
    stringstyle=\color{mauve},
    breaklines=true,
    breakatwhitespace=true,
    escapeinside=``,%逃逸字符(1左面的键),用于显示中文例如在代码中`中文...`
    tabsize=4,
    extendedchars=false %解决代码跨页时,章节标题,页眉等汉字不显示的问题
}


\hypersetup{colorlinks=true,linkcolor=cyan,anchorcolor=megenta,citecolor=yellow}
\geometry{right=2.5cm,left=2.5cm,top=2.5cm,bottom=2.5cm}
\renewcommand{\headrulewidth}{0pt}%去掉页眉分划
\pagestyle{fancy}

% ------------------文字底色------------------
\newtheorem{example}{}
\newtheorem{examp}{}
\definecolor{egshade}{rgb}{0.96,0.96,0.96}% 
\newenvironment{eg}{%
\def\FrameCommand{%
\hspace{1pt}%
{\color{Gray}\vrule width 2pt}%竖线颜色
{\color{egshade}\vrule width 4pt}%
\colorbox{egshade}%
}%
\MakeFramed{\advance\hsize-\width\FrameRestore}%
\noindent\hspace{-4.55pt}% 
\begin{adjustwidth}{}{7pt}%
\vspace{2pt}\vspace{2pt}%
\normalfont %环境字体设置
}
{%
\vspace{2pt}\end{adjustwidth}\endMakeFramed%
}

\newtheorem{warning}{}  
\definecolor{washade}{rgb}{0.99,0.95,0.94} % 方框底部颜色
\newenvironment{wa}{%
\def\FrameCommand{%
\hspace{1pt}%
{\color{LightCoral}\vrule width 2pt}%竖线颜色
{\color{washade}\vrule width 4pt}%
\colorbox{washade}%
}%
\MakeFramed{\advance\hsize-\width\FrameRestore}%
\noindent\hspace{-4.55pt}% disable indenting first paragraph
\begin{adjustwidth}{}{7pt}%
\vspace{2pt}\vspace{2pt}%
\normalfont %
}
{%
\vspace{2pt}\end{adjustwidth}\endMakeFramed%
}



% ----------------------正文------------------------------------------------------------
\begin{document} 
\centerline{\LARGE \textbf{数学软件} }\par 

% ---------------23.02.21----------------
\noindent \Large \section*{23.02.21} \par \normalsize
\begin{eg}
    1.计算$\sum \limits _{i=1}^{100} 2^{i}$ \par
\end{eg}
\begin{lstlisting}
    i = 1 : 100
    a = 2.^i    % i 是向量,所以要用点乘
    sum(a)
\end{lstlisting}


\begin{eg}
    2.将数取整的几种方式
\end{eg}
\begin{lstlisting}
    a = 5.3
    floor(a)   %向下取整
    ceil(a)    %向上取整
    round(a)    %四舍五入
\end{lstlisting}

\begin{eg}
    3.自定义函数
\end{eg}
\begin{lstlisting}
    function y = gab(x)
    y = log2(abs(x^2 - 5)) + (sin(x) + cos(x))^6; 
    %新建一个 gab.m 文件将函数自定义,在命令行窗口可直接求得gab(x)的值
\end{lstlisting}
\begin{lstlisting}
    function y = gaa(x)
    y = 100*(x(2) - x(1))^2 + x(1)*x(2);
    %两个自变量的情况,命定行窗口gaa([x1,x2])
\end{lstlisting}

\begin{eg}
    4.取值问题
\end{eg}
\begin{lstlisting}
    i = 4 : 2 : 10  %从4到10,步长为2的所有数
    linspace(1 ,10 ,5)  %从1到10等长取5个数
    logspace(1 ,1024 ,10)   %创建行向量,从10^1到10^1024,生成总数为10的等比数列
\end{lstlisting}

\begin{eg}
    5.矩阵的常见使用函数
\end{eg}
\begin{lstlisting}
    a = [1 2 3 ;4 5 6 ;7 8 9]
    a'  %转置
    a( : , 3) = [ ]   %去除第3列
    a( 2, : ) = [ ]   %去除第2行
    det(a)  %求行列式
    inv(a)  %求逆
    rank(a)  %求秩
    format rat  %用分数形式表示数值
\end{lstlisting}
\par

% -----------------23.02.24------------------
\noindent \Large \section*{23.02.24} \par \normalsize
\begin{eg}
    1.输入矩阵
\end{eg}
\begin{lstlisting}
    %----常规----
    ones(m,n)   %全1
    zeros(m,n)   %全0
    eye(m,n)    %单位矩阵
    %----拼接----
    >> C = [A B]    %行

C =

     1     1     0     0
     1     1     0     0

    >> C = [A ;B]   %列
 
C =

     1     1
     1     1
     0     0
     0     0

    %----直接输入----
    a = [1 2 3 ;4 5 6 ;7 8 9]
\end{lstlisting}

\begin{eg}
    2.数组的运算

    当两个数组具有相同的维数时,运算可按元素对元素的方式进行,不同大小或维数的数组不能进行运算P248
\end{eg}

\begin{eg}
    3.空矩阵的作用
\end{eg}
\begin{lstlisting}
    %----去掉某行、列----
    a = [1 2 3 ;4 5 6 ;7 8 9]
    a( : , 3) = [ ]   %去除第3列
    a( 2, : ) = [ ]   %去除第2行  
\end{lstlisting}

\begin{eg}
    4.矩阵中元素的操作
\end{eg}
\begin{lstlisting}
    %----提取某行、列----
    a(1 : 2, : )    %提取1、2行,所有列
    a(2 : 2 : 100 , : )    %提取偶数行
    a( : , 1 : 2 : 99)    %提取奇数行
    a( : , 99 : -2 : 1)    %提取奇数行逆序
\end{lstlisting}

\begin{eg}
    5.矩阵的运算
\end{eg}
\begin{lstlisting}
    A + B   %矩阵加法
    A * B   %矩阵惩罚
    det(A)  %方阵的行列式
    inv(A)  %方阵的逆
    rank(A)  %求秩
    eig(A)  %方阵特征值与特征向量
\end{lstlisting}

\begin{wa}
    行列式的原始定义

    \noindent 特征值、特征向量的代数和几何意义
\end{wa}

\begin{eg}
    6.运算符,详见P248
\end{eg}

\begin{eg}
    6.控制流结构
\end{eg}

\begin{wa}
    例:计算$\sum \limits _{i=1}^{100} i$ 
\end{wa}
\begin{lstlisting}
    %----for----
    s = 0
    for i = 1 : 100
        s = s + i
    end
    %----while----
    s = 0 , i = 1
    while i <= 100
        s = s + i
        i = i + 1
    end
\end{lstlisting}

\begin{wa}
    例:$1.005^{n} \geqslant 2$,求$n$的最小值
\end{wa}
\begin{lstlisting}
    %----teacher----
    n = 0   %这里取n = 1也不影响结果
    while 1.005^n < 2 
        n = n + 1
    end

    %----me----
    n = 0
    while n < log(2)/log(1.005)
        n = n + 1
    end
\end{lstlisting}

\begin{wa}
    例:计算$\sum \limits _{i=1}^{100} \dfrac{6^{i - 1}}{7^{C(i)}}$
    $$C(i) =
    \begin{cases}
        2i & i \leqslant 5 \\
        i^{2} & i > 5
    \end{cases}
    $$
\end{wa}
\begin{lstlisting}
    %----if-else----
    s = 0 
    for i = 1 : 100
        if i <= 5
            t = 6^( i - 1 ) / 7^( 2*i ) 
        else
            t = 6^( i - 1 ) / 7^( i^2 ) 
        end
            s = s + t 
    end
\end{lstlisting}

\begin{wa}
    例:设
    $$f(x) = 
    \begin{cases}
        x^{2} + 1 & x > 1\\
        2x & x \leqslant 1
    \end{cases}
    $$
    \indent 求$f(2)$,$f(1)$
\end{wa}
\begin{lstlisting}
    %----新建M文件----
    function y = fun1(x)
    if x > 1
        y = x^2 + 1
    else
        y = 2*x
    end
\end{lstlisting}

\begin{wa}
    例:设
    $$f(x) = 
    \begin{cases}
        x^{2} + 1 & x > 1\\
        2x & 0 < x \leqslant 1 \\
        x^{3} & x \leqslant 0
    \end{cases}
    $$
    \indent 求$f(2)$,$f(0.5)$,$f(-1)$
\end{wa}
\begin{lstlisting}
    %----新建M文件----
    function y = fun1(x)
    if x > 1
        y = x^2 + 1
    elseif 0 < x <= 1
        y = 2 * x
    else
        y = x^3
    end
\end{lstlisting}

\begin{wa}
    作业
\end{wa}
\begin{lstlisting}
    function y = sort(x)
    for i = 1 : 9
        for j = 1 : 10-i
            if x( j ) > x( j + 1 )
                a = x( j );
                x( j ) = x( j + 1 ); 
                x( j + 1 ) = a ;
            end
        end
        y = x ;
    end
\end{lstlisting}
\begin{lstlisting}
    function m =maxele( a ) 
    ii = 1 ;
    jj = 1 ;
    i = 1 ; j = 1 ; 
    m = a( i  ,  j ) ;
    for i = 1 : 4   
        for j = 1 : 5
            if m < a( i , j )
                m =  a( i , j ) ;
                ii = i ;
                jj = j ;
            end
        end
    end
    ii
    jj
    end
\end{lstlisting}
\begin{lstlisting}
    function y = fac(x)
    a = 1;
    y = 0;
    for n = 1 : x
            a = a * n ;
            y = y + a ;      
    end 
    end
\end{lstlisting}
\begin{lstlisting}
    function y = balls(x)
    y = 0 ;
    s = 100 ;
    for a = 1 : x
        s = s / 2 ;
        y = y + 2*s + s ;
    end
    s
\end{lstlisting}
\begin{lstlisting}
    function y = cula (a,b)
    y = a^2 + sin(a*b) + 2*b ;
\end{lstlisting}



% -----------------23.02.28------------------
\noindent \Large \section*{23.02.24} \par \normalsize
\begin{eg}
    "作业1"习题讲解
\end{eg}
\begin{lstlisting}
    function y = sorted1( x )
    a = randperm( x );
    n = length( a ) ;
    for i = 1 : n - 1
        for j = n : -1 : i + 1    %起泡法从后往前排序
            if a( j ) < a( j - 1 )
                b = a( j - 1 ) ;
                a( j - 1 ) = a( j ) ;
                a( j ) = b ;
            end
            y = a
        end
    end
\end{lstlisting}
\begin{lstlisting}
    function y = maxele1( m , n )
    a = rand( m , n ) ;
    ma = -inf ; w1 = [ ] ;
    for i = 1 : m
        for j = 1 : n
            if a( i , j ) >= ma
                ma = a( i , j ) ;
                w = [ i , j ] ;
                w1 = [ w1 ; w ] ;   %当有相同元素出现多个坐标的常见处理方法
            end
        end
    end
    y = ma ;
\end{lstlisting}

\begin{eg}
    1.赋值语句
\end{eg}
\begin{lstlisting}
    for i = 1 : 10
        y = sin( i * pi / 180 ) ;
    end
\end{lstlisting}

\begin{eg}
    2.内存变量的删除和修改
\end{eg}
\begin{lstlisting}
    save big.mat    %储存
    load big.mat    %调用
\end{lstlisting}

\begin{eg}
    3.常用数学函数
\end{eg}
\begin{lstlisting}
    sin(n * pi / 180)  %弧度转换
    abs(n)    %实数的绝对值、复数的模、字符串的ASCII码值
    fix(n)    %函数取整
    floor(n)
    ceil(n)
    round(n)
    rem(x , y)    %当x 、y不同号,结果不同
    mod(x , y)    %x、y必须为相同大小的实矩阵或为标量
\end{lstlisting}

\begin{eg}
    4.矩阵元素
\end{eg}
\begin{lstlisting}
    A(3 , 2) = 200    %通过下标引用矩阵元素
    A = [ 1 , 2 , 3 ; 4 , 5 ,6 ] ;      %采用矩阵元素序号引用元素
    A(3)
    ans = 2     %在MATLAB中,列优先
\end{lstlisting}

\begin{eg}
    5.矩阵拆分
\end{eg}
\begin{lstlisting}
    A( : , j )      %取第j列全部元素
    A( i , : )      %取第i行全部元素
    A( i : i + m , : )      %第i到第i+m行全部元素
    A( : , k : k + m)       %第k到第k+m列全部元素
    A( 2 : end , : )    %第2到最后一行全部元素
\end{lstlisting}

\begin{eg}
    6.特殊矩阵
\end{eg}
\begin{lstlisting}
    zeros(n)    %零矩阵
    ones(n)   %幺矩阵
    eye(n)  %单位阵
    rand(n) %0~1间随机均匀分布矩阵
    randn(n)    %产生均值为0,方差为1的标准正态分布矩阵
    reshape(A , m , n)  %将矩阵A重排为m*n
\end{lstlisting}

\begin{wa}
    建立随机矩阵:
    
    (1)在区间$[20,50]$内均匀分布的$5$阶随机矩阵
    
    (2)均值为0.6,方差为0.1的5阶正态分布随机矩阵
\end{wa}
\begin{lstlisting}
    x = 20 + (50 - 20)*rand(5)
    y = 0.6 + sqrt(0.1)*randn(5)
\end{lstlisting}

\begin{eg}
    7.魔法阵、$Vandermonde$阵、$Pascal$阵
\end{eg}

\begin{wa}
    将101~125填入一个5行5列的表格中,使其每行每列及对角线的和均为565
\end{wa}
\begin{lstlisting}
    M = 100 + magic(5)
\end{lstlisting}

\begin{eg}
    8.对角阵、三角阵
\end{eg}


\end{document}